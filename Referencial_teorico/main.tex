
\subsection{Controladores PID e Sistemas Dinâmicos de Primeira Ordem}
%\subsection{Sistemas de Primeira Ordem}\hspace{4ex}

\hspace{4ex}O controlador PID, na realidade, pode ser visto como sendo uma família de controladores. As 3 ações de controle podem ser combinadas de diferentes maneiras, formando controladores que podem receber diferentes nomes, como: P, PI, PD, PID, conforme as ações de controle são, ou não, utilizadas.

\begin{enumerate}
    \item Controle Proporcional (P):
    \[u(t)=Kpe(t)\] 
    \[U(S)=KpE(S)\]
        \begin{itemize}
            \item O controlador proporcional é um amplificador, com ganho ajustável (K);
            \item O aumento do ganho K, diminui o erro de regime;
            \item Em geral, o aumento de K torna o sistema mais oscilatório, podendo instabilizá-lo;
            \item Melhora o regime e piora o transitório, sendo bastante limitado.
        \end{itemize}
        
    \item Controle Proporcional + Integral (PI):
    \[u(t)=Kp(e(t)+\frac{1}{\tau i}\int_{0}^{t}e(\tau)d\tau)\]
    \[U(S)=\frac{(KpS+Ki)}{s}E(s)\]
        \begin{itemize}
            \item A ação integral do controlador move a variável de controle C(S) baseada na integral no tempo do erro, sendo: \[Ki=\frac{Kp}{\tau i}\], em que: \[\tau i\] é o tempo integrativo, ou tempo de \emph{reset}, com unidade da ordem de minutos;
            \item Zera o erro de regime, pois aumenta o tipo do sistema em 1 unidade;
            \item É utilizado quanto tempos resposta transitória aceitável e resposta em regime insatisfatória;
            \item Adiciona um pólo em p=0 e um zero em \[z=\frac{ki}{kp}=\frac{-1}{\tau i}\]
            \item Como aumenta a ordem do sistema, temos possibilidade de instabilidade diferente do sistema original. Pode degradar o desempenho do controlador em malha fechada.
        \end{itemize}
        
    \item Controle Proporcional + Derivativo (PD):
        \[u(t)=Kp(e(t)+\tau d\frac{de(t)}{dt})\]
        \[U(S)=(Kp+KdS)E(S)\]
        \begin{itemize}
            \item Sendo: \[Kd=Kp\tau d\] a constante derivativa, dada em minutos.
            \item Leva em conta a taxa de variaçao do erro;
            \item É utilizado quando temos resposta em regime aceitável e resposta transitória insatisfatória;
            \item Adiciona um zero em \[z=\frac{-Kp}{Kd}=\frac{-1}{\tau d}\]
            \item Introduz um efeito de antecipação no sistema, fazendo com que o mesmo reaja não somente à magnitude do sinal de erro, como também à sua tendência para o instante futuro, iniciando, assim, uma ação corretiva mais cedo;
            \item A ação derivativa tem a desvantagem de amplificar os sinais de ruído, o que pode causar um efeito de saturação nos atuadores do sistema.
        \end{itemize}
        
    \item Controle Proporcional + Integral + Derivativo (PID):
        \[u(t)=Kp(e(t)+\frac{1}{\tau i}\int_{0}^{t}e(\tau)d(\tau)+\tau d\frac{de(t)}{dt})\]
        \[U(S)=(Kp+\frac{Ki}{s}+KdS)E(S) => \frac{U(S)}{E(S)}=\frac{KdS^2+KpS+Ki}{S}\]
        \begin{itemize}
                    \item É utilizado quando temos resposta transitória e em regime insatisfatórias simultaneamente;
            \item Adiciona um pólo em p=0 e 2 zeros, que dependem dos parâmetros do controlador.
        \end{itemize}
        
    \item Implementação do Controlador PID:
        \begin{itemize}
            \item Diferentes equipamentos, de diferentes fabricantes, podem apresentar pequenas variações quanto à implementação do controlador PID. As duas implementações mais comumente utilizadas são: \textbf{Ideal} e \textbf{Paralelo}.
        \end{itemize}
        
    \item Modificações das Ações de Controle PID:
        \begin{itemize}
            \item Existem também diversas modificações, que poderão ser necessárias em cada implementação, dependendo de características do sistema que estiver sendo controlado, das condições de operações as quais o sistema será submetido e, até mesmo, do equipamento que será utilizado para implementação do controlador.
        \end{itemize}
        
        \begin{enumerate}
            \item Modificação na Ação Derivativa:
                \begin{itemize}
                    \item Filtro da Ação Derivativa
                        \begin{figure}[h]
                            \centering
                            \includegraphics[width=15cm]{images/roteiro a/img ref teorico/filtro_de_acao_derivativa.png}
                            \caption{Filtro de Ação derivativa}
                            \label{fig:filtro_de_acao_derivativa}
                        \end{figure}
                    \item PI-D
                        \begin{figure}[h]
                            \centering
                            \includegraphics[width=15cm]{images/roteiro a/img ref teorico/pid.png}
                            \caption{PI-D}
                            \label{fig:pid}
                        \end{figure}
                        \begin{itemize}
                            \item Objetivo: Não derivar variações bruscas no sinal de referência. 
                        \end{itemize}
                        \newpage
                    \item I-PD
                         \begin{figure}[h]
                            \centering
                            \includegraphics[width=15cm]{images/roteiro a/img ref teorico/ipd.png}
                            \caption{I-PD}
                            \label{fig:ipd}
                        \end{figure}
                        \begin{itemize}
                            \item Objetivo: Não derivar, nem amplificar variações bruscas no sinal de referência. 
                            \\
                        \end{itemize}
                \end{itemize} 
            \item Modificação na Ação Integrativa:
                \begin{itemize}
                    \item Filtro Anti-Windup (Anti-Reset Windup)
                \end{itemize}
                    \begin{figure}[h]
                            \centering
                            \includegraphics[width=15cm]{images/roteiro a/img ref teorico/filtro_anti_windup.png}
                            \caption{Filtro Anti-Windup (Anti-Reset Windup)}
                            \label{fig:filtro_anti_windup}
                        \end{figure}
        \end{enumerate}
\end{enumerate}
\newpage

\subsection{Sistemas de Segunda Ordem}
\hspace{4ex}Considere a seguinte equação diferencial de segunda ordem:
    \[a\frac{d^2e(t)}{dt^2}+b\frac{de(t)}{dt}+dc(t)=er(t)\]
Definindo:
    \[\frac{b}{a}=2\epsilon\omega_n; \hspace{4ex} \frac{d}{a}=\omega^2_n; \hspace{4ex} \frac{e}{a}=K;\]
onde \(\epsilon\) é o fator de amortecimento, \(\omega_n\) é a frequência natural e K é o ganho do sistema, temos:
    \[\frac{d^2c(t)}{dt^2}+2\epsilon\omega_n\frac{dc(t)}{dt}+\omega_n^2c(t) = Kr(t)\]
Aplicando Laplace com C.I. nulas: 
    \[\frac{C(s)}{R(s)} = \frac{K}{S^2+2\epsilon\omega_nS+\omega_n^2}\]

\subsection{Sistemas de Segurança Instrumentados ou Intertravamentos}\hspace{4ex}

\subsection{Sistemas Dinâmicos na Configuração de Controle em Cascata}\hspace{4ex}