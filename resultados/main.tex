\newcommand{\labABSubFigure}[4]{
% o tipo do sistema 1 ou 2
% tipo do controlador P
% final do arquivo -> #3
% titulo
\ifnum#1=1
    \def \roteiro {roteiro a}
    \def \sistema {sistema tipo 1 }
\fi
\ifnum#1=2
    \def \roteiro {roteiro b}
    \def \sistema {sistema tipo 2 }
\fi
%
\def \controlador{controlador #2}
\def \ganho{#3.pdf}
\def \caminhoImagem{images/\roteiro/\controlador/\ganho}
%
%
\begin{subfigure}[b]{.5\textwidth}
  \centering
  \includegraphics[width=.8\linewidth]{\caminhoImagem}
  \caption{\sistema $#4$}
\end{subfigure}%
}

\newcommand{\labMasterSlaveSubFigure}[4]{
% o tipo do sistema 1 ou 2
% tipo do controlador P
% final do arquivo -> #3
% titulo
\def \controladorMestre{controlador Mestre #1}
\def \controladorEscravo{Controlador escravo #2}
\def \ganho{#3.pdf}
\def \caminhoImagem{images/roteiro c/\controladorMestre/\controladorEscravo/\ganho}
%
%
\begin{subfigure}[b]{.5\textwidth}
  \centering
  \includegraphics[width=.8\linewidth]{\caminhoImagem}
  \caption{\controladorEscravo \\\hspace{.5\textwidth} $#4$}
\end{subfigure}%
}




resultados obtidos do simulador e discutir eles
explicar como essa seção foi dividida



\subsection{Controlador P}\hspace{4ex}

\begin{figure}[h]
\labABSubFigure{1}{P}{kp_5}{K_p=5}%1-1
\labABSubFigure{2}{P}{kp_10}{K_p=10}%2-1
\labABSubFigure{1}{P}{kp_10}{K_p=10}%1-2
\labABSubFigure{2}{P}{kp_20}{K_p=20}%2-2
\labABSubFigure{1}{P}{kp_20}{K_p=20}%1-3
\labABSubFigure{2}{P}{kp_100}{K_p=100}%2-3
\end{figure}

Testes de controle de primeira ordem, para entrada constante de valor 15 e senoidal, de amplitude 2 e bias 2.

Pode-se observar que, no sistema de tipo 1, com o aumento do ganho proporcional houve um aumento na velocidade do sistema até a resposta em regime permanente e uma diminuição do erro estacionário, ou seja, a diferença entre a resposta desejada e a resposta em regime permanente.

Observou-se que também que, no sistema de tipo 2, o aumento de $K_p$ aumentou a oscilação do sistema mas diminuiu o erro de regime, não zerando-o completamente, como esperado. Sendo assim, o sistema é do tipo subamortecido (0 < $\epsilon$ < 1), e considerando que $K_p$ = $\omega_n^2$, a frequência aumenta com o ganho. Houve aumento no tempo de acomodação e haveria também aumento significativo no \emph{overshoot}, como se pode observar pela resposta do controlador que é quase uma onda quadrada em $K_p$ = 100. Não ocorreu devido a saturação do sistema.

\newpage

\subsection{Controlador PI}\hspace{4ex}
\begin{figure}[h]
\labABSubFigure{1}{PI}{kp_20_ki_0_5}{K_p=20 \quad K_i=0.5}%1-1
\labABSubFigure{2}{PI}{kp_20_ki_0_01}{K_p=20 \quad K_i=0.01}%2-1
\labABSubFigure{1}{PI}{kp_20_ki_5}{K_p=20 \quad K_i=5}%1-2
\labABSubFigure{2}{PI}{kp_20_ki_0_5}{K_p=20 \quad K_i=0.5 }%2-2
\labABSubFigure{1}{PI}{kp_20_ki_10}{K_p=20 \quad K_i=10}%1-3
\labABSubFigure{2}{PI}{kp_20_ki_0_005}{K_p=20 \quad K_i=0.005}%2-3
\end{figure}

No sistema tipo 1, com a adição do controlador integral o sistema passa a ter comportamento de segunda ordem, subamortecido no exemplo, zerando o erro de regime observado anteriormente. No entanto, há bastante oscilação e o sistema é mais lento, chegando ao regime permanente em cerca de 3 vezes o tempo de sua contraparte apenas proporcional. Também há o custo do \emph{overshoot} de aproximadamente 36\% quando $K_i$ = 10, como observado.

No sistema tipo 2, o erro de regime é automaticamente zerado, no entanto o intertravamento precisa agir mesmo para valores diminutos do ganho integral $K_i$ em 0.5, para que o tanque 1 não transborde (passe do nível 29). O efeito cumulativo da parcela integral do controlador, quando tem seu ganho aumentado, gera uma resposta mais oscilatória e com maior \emph{overshoot}, até que instabiliza o sistema em $K_i$ muito altos. Nesta situação o sistema continua a oscilar sem nunca chegar ao regime permanente em tempo plausível. Essa tendencia foi observada no experimento.

\newpage

\subsection{Controlador PD}\hspace{4ex}

\begin{figure}[h]
\labABSubFigure{1}{PD}{kp_20_kd_1}{K_p=20 \quad K_d=1}%1-1
\labABSubFigure{2}{PD}{kp_20_kd_1}{K_p=20 \quad K_d=1}%2-1
\labABSubFigure{1}{PD}{kp_20_kd_5}{K_p=20 \quad K_d=5}%1-2
\labABSubFigure{2}{PD}{kp_20_kd_15}{K_p=20 \quad K_d=15 }%2-2
\labABSubFigure{1}{PD}{kp_20_kd_15}{K_p=20 \quad K_d=15}%1-3
\labABSubFigure{2}{PD}{kp_20_kd_25}{K_p=20 \quad K_d=25}%2-3
\end{figure}

A ação derivativa nesse contexto tem função de antecipar a ação de controle, ao levar em conta a “inércia” do erro, gerando maior estabilidade relativa e velocidade de resposta transitória. Com isso se observou o aumento do erro relativo a medida em que se aumentada o ganho diferencial $K_d$. Também é atestado a amplificação do ruído, já que o mesmo é formado por bruscas variações em curtos períodos de tempo. Esse ruído amplificado, no entanto, é saturado em -4/+4.

Já no sistema 2, ao aumentar o ganho diferencial $K_d$ o \emph{overshoot} foi atenuado em aproximadamente 10\% de $K_d$ = 1 para $K_d$ = 25, enquanto a sensibilidade do controlador as alterações advindas do ruído foram bastante acentuadas. Além disso, nesse contexto, a velocidade da resposta transitória foi praticamente dobrada (tempo de acomodação cai pela metade).


\newpage

\subsection{Controlador PID}\hspace{4ex}
\begin{figure}[h]
\labABSubFigure{1}{PID}{kp_20_ki_0_5_kd_22}{K_p=20 \quad K_i=0.5 \quad K_d=22}%1-1
\labABSubFigure{2}{PID}{kp_20_ki_1_kd_250}{K_p=20 \quad K_i=1 \quad K_d=250}%2-1
\labABSubFigure{1}{PID}{kp_20_ki_1_kd_22}{K_p=20 \quad K_i=1 \quad K_d=22}%1-2
\labABSubFigure{2}{PID}{kp_20_ki_1_kd_500}{K_p=20 \quad K_i=1 \quad K_d=500 }%2-2
\labABSubFigure{1}{PID}{kp_20_ki_2_kd_22}{K_p=20 \quad K_i=2 \quad K_d=22}%1-3
\labABSubFigure{2}{PID}{kp_20_ki_1_kd_1000}{K_p=20 \quad K_i=1 \quad K_d=100}%2-3
\end{figure}

Nesse contexto, o controlador proporcional ajusta a variável de controle proporcionalmente ao erro, o integral ajusta proporcional ao erro acumulado e a derivativa proporcional a velocidade de variação do erro. Permite um maior controle do comportamento do sistema, em relação aos anteriores. Observou-se um aumento no \emph{overshoot} ao se aumentar $K_i$ e uma maior oscilação com aumento de $K_p$.

Como observado no controlador PI, instabilização ocorre para maiores valores de $K_i$, onde o sistema nunca alcança regime permanente. O mesmo efeito ocorre ao se alterar $K_p$ mas não na mesma proporção. Alterar $K_d$ afeta notavelmente o tempo de acomodação e a sensibilidade do controlador ao efeito do ruído, como visto no do tipo PD. Para valores menores de $K_d$ (notadamente $K_d$ = 100) a frequência diminui de tal maneira a aumentar o tempo de acomodação. 


\newpage

\subsection{Controlador PID com filtro na ação derivativa}\hspace{4ex}

\def \PIDDerivativeFilter{PID filtro derivativo}

\begin{figure}[h]
\labABSubFigure{1}{\PIDDerivativeFilter}{kp_20_ki_1_kd_1_kn_0_5}{K_p=20 \quad K_i=1 \quad K_d=1 \quad K_n=0.5}%1-1
\labABSubFigure{2}{\PIDDerivativeFilter}{kp_20_ki_1_kd_1_kn_30}{K_p=20 \quad K_i=1 \quad K_d=1 \quad K_n=30}%2-1
\labABSubFigure{1}{\PIDDerivativeFilter}{kp_20_ki_1_kd_1_kn_50}{K_p=20 \quad K_i=1 \quad K_d=1 \quad K_n=50}%1-2
\labABSubFigure{2}{\PIDDerivativeFilter}{kp_20_ki_1_kd_1_kn_50}{K_p=20 \quad K_i=1 \quad K_d=1 \quad K_n=50}%2-2
\labABSubFigure{1}{\PIDDerivativeFilter}{kp_20_ki_1_kd_1_kn_100}{K_p=20 \quad K_i=2 \quad K_d=1 \quad K_n=100}%1-3
\labABSubFigure{2}{\PIDDerivativeFilter}{kp_20_ki_1_kd_1_kn_100}{K_p=20 \quad K_i=1 \quad K_d=1 \quad K_n=100}%2-3
\end{figure}

O controlador derivativo está suscetível a amplificar ruído de baixa amplitude e alta frequência, podendo comprometer o sistema. É possível criar um laço de realimentação com uma integral no retorno e ganho N, que age como um filtro passa-baixas, atenuando o efeito do ruído, como se pode observar ao comparar a amplitude de oscilações com e sem filtro.

Em ambos sistemas observou-se eliminação do \emph{overshoot} e aumento no tempo de acomodação para valores altos de $K_n$ (notadamente $K_n$ $\geq$ 100). Além disso a onda passou a "deformar" já que nessa situação se coloca muito peso na resposta influenciada por ruído aleatório.

\newpage

\subsection{Controlador PID com filtro anti-reset-windup}\hspace{4ex}

\def \PIDFilter{PID filtro windup}

\begin{figure}[h]
\labABSubFigure{1}{\PIDFilter}{kp_0_5_ki_0_5_kd_0_05_kaw_0_5}{K_p=0.5 \quad K_i=0.5 \quad K_d=0.05 \quad K_{aw}=0.5}%1-1
\labABSubFigure{2}{\PIDFilter}{kp_20_ki_1_kd_1_kaw_1}{K_p=20 \quad K_i=1 \quad K_d=1 \quad K_{aw}=1}%2-1
\labABSubFigure{1}{\PIDFilter}{kp_0_5_ki_0_5_kd_0_05_kaw_1}{K_p=0.5 \quad K_i=0.5 \quad K_d=0.05 \quad K_{aw}=1}%1-2
\labABSubFigure{2}{\PIDFilter}{kp_20_ki_1_kd_20_kaw_1}{K_p=20 \quad K_i=1 \quad K_d=20 \quad K_{aw}=1}%2-2
\labABSubFigure{1}{\PIDFilter}{kp_0_5_ki_1_kd_0_05_kaw_0_5}{K_p=0.5 \quad K_i=1 \quad K_d=0.05 \quad K_{aw}=0.5}%1-3
\labABSubFigure{2}{\PIDFilter}{kp_20_ki_5_kd_1_kaw_1}{K_p=20 \quad K_i=5 \quad K_d=1 \quad K_{aw}=1}%2-3
\end{figure}

Quando o sistema está sujeito a um erro constante diferente de zero, o controlador integral tende a acumular seu efeito e requisitar comandos cada vez maiores ao atuador, mesmo estando saturado. Quando o erro eventualmente decresce e se torna negativo, há um tempo até que a saída integral “retorne” ao seu estado funcional, é o que se chama de tempo de \emph{windup}. O filtro \emph{anti-windup} contrabalanceia esse efeito somando ou subtraindo um valor proporcional ao ganho $K_aw$ quando há saturação (saída do controlador e entrada da planta são diferentes).

O filtro \emph{anti-windup} atenua o efeito da parcela integral do controlador. Com $K_aw$ = 0.05 há uma leve redução do \emph{overshoot} e do tempo de acomodação. A frequência natural não foi modificada ao aumentar $K_aw$, apenas a mesma redução citada anteriormente, \emph{overshoot} por exemplo cai de 50\% para 35\% e tempo de acomodação (aproximadamente 5\%) de 180 ms para 130 ms aumentando o ganho do filtro de 0.05 para 0.5. O \emph{overshoot} do tanque 2 é completamente removido para valores menores de $K_i$.


\newpage
%

%
\subsection{Sistema de Controle Mestre-Escravo: Mestre P}

\subsubsection{Controlador Escravo P}
\begin{figure}[h]
\foreach \kp in {40,300,500}{
 \labMasterSlaveSubFigure{P}{P}{kp_master_\kp_kp_slave_\kp}
    {K_{p_{\small{mestre}}}=\kp \quad K_{p_{\small{escravo}}}=\kp }%
}
\end{figure}

\newpage

\subsubsection{Controlador Escravo PD}

\begin{figure}[h]
\foreach \kdSlave in {100,1000,1000}{
    \labMasterSlaveSubFigure{P}{PD}{Kp_master_500_Kp_slave_500_Kd_slave_\kdSlave}
    {K_{p_{\small{mestre}}}=500 \quad K_{p_{\small{escravo}}}=500%
    \quad K_{d_{\small{escravo}}}=\kdSlave}%
    }
\end{figure}

\newpage

\subsubsection{Controlador Escravo PI}

\begin{figure}[h]
\foreach \kiSlave in {3,6,12}{
    \labMasterSlaveSubFigure{P}{PI}{Kp_master_500_Kp_slave_500_Ki_slave_\kiSlave}
    {K_{p_{\small{mestre}}}=500 \quad K_{p_{\small{escravo}}}=500%
    \quad K_{i_{\small{escravo}}}=\kiSlave}%
    }
\end{figure}

\newpage


\subsubsection{Controlador Escravo PID}

\begin{figure}[h]
    \foreach \ki/\kd in {0.5/5000,1/300,3/300}{
    \labMasterSlaveSubFigure{P}{PID}
    {Kp_master_500_Kp_slave_500_Ki_slave_\ki_Kd_slave_\kd}
    {K_{p_{\small{mestre}}}=500 \quad K_{p_{\small{escravo}}}=500%
    \quad K_{i_{\small{escravo}}}=\ki \quad K_{d_{\small{escravo}}}=\kd}%
    }

\end{figure}

\newpage

\subsubsection{Controlador Escravo PID filtro derivativo}

\begin{figure}[h]
  \foreach \ki/\kd/\kn in {1/100/2,1/100/15,7/100/15}{
      \labMasterSlaveSubFigure{P}{PID filtro derivativo}
    {Kp_master_500_Kp_slave_500_Ki_slave_\ki_Kd_slave_\kd_Kn_slave_\kn}
    {K_{p_{\small{mestre}}}=500 \quad K_{p_{\small{escravo}}}=500%
    \quad K_{i_{\small{escravo}}}=\ki \quad K_{d_{\small{escravo}}}=\kd%
    \quad K_{n_{\small{escravo}}}=\kn }
    }
\end{figure}


\newpage

\subsubsection{Controlador Escravo PID filtro windup}

\begin{figure}[h]
  \foreach \ki/\kd/\kaw in {100/100/1,300/100/2,2000/100/3}{
      \labMasterSlaveSubFigure{P}{PID filtro windup}
    {Kp_master_500_Kp_slave_500_Ki_slave_\ki_Kd_slave_\kd_Kaw_slave_\kaw}
    {K_{p_{\small{mestre}}}=500 \quad K_{p_{\small{escravo}}}=500%
    \quad K_{i_{\small{escravo}}}=\ki \quad K_{d_{\small{escravo}}}=\kd%
    \quad K_{aw_{\small{escravo}}}=\kaw }
    }
\end{figure}
%
\newpage
%
\subsection{Sistema de Controle Mestre-Escravo: Mestre PD}

\subsubsection{Controlador Escravo P}
\begin{figure}[h]
\foreach \kd in {500,1000,10000}{ %kp_master_500_kd_master_500_kp_slave_500
 \labMasterSlaveSubFigure{PD}{P}{kp_master_500_kd_master_\kd_kp_slave_500}
    {K_{p_{\small{mestre}}}=500 \quad K_{d_{\small{mestre}}}=\kd \quad K_{p_{\small{escravo}}}=500 }%
}
\end{figure}

\newpage

\subsubsection{Controlador Escravo PD}

\begin{figure}[h]
\foreach \kdMaster/\kdSlave in {1/1,1/100,5/100}{
    \labMasterSlaveSubFigure{PD}{PD}
    {Kp_master_500_Kd_master_\kdMaster_Kp_slave_500_Kd_slave_\kdSlave}
    {K_{p_{\small{mestre}}}=500 \quad K_{d_{\small{mestre}}}=\kdMaster
    \quad K_{p_{\small{escravo}}}=500 \quad K_{d_{\small{escravo}}}=\kdSlave}%
    }
\end{figure}

\newpage

\subsubsection{Controlador Escravo PI}

\begin{figure}[h]
\foreach \kdMaster/\kiSlave in {1/10,1/50,10000/10}{
    \labMasterSlaveSubFigure{PD}{PI}
    {Kp_master_500_Kd_master_\kdMaster_Kp_slave_500_Ki_slave_\kiSlave}
    {K_{p_{\small{mestre}}}=500 \quad K_{d_{\small{mestre}}}=\kdMaster \quad K_{p_{\small{escravo}}}=500 \quad K_{i_{\small{escravo}}}=\kiSlave}%
    }
\end{figure}

\newpage


\subsubsection{Controlador Escravo PID}

\begin{figure}[h]
    \foreach \ki/\kd in {0.5/5000,1/300,3/300}{
    \labMasterSlaveSubFigure{PD}{PID}
    {Kp_master_500_Kd_master_1_kp_slave_500_ki_slave_10_kd_slave_500}
    {K_{p_{\small{mestre}}}=500 \quad K_{p_{\small{escravo}}}=500%
    \quad K_{i_{\small{escravo}}}=\ki \quad K_{d_{\small{escravo}}}=\kd}%
    }

\end{figure}

\newpage

\subsubsection{Controlador Escravo PID filtro derivativo}

\begin{figure}[h]
  \foreach \ki/\kd/\kn in {1/100/2,1/100/15,7/100/15}{
      \labMasterSlaveSubFigure{P}{PID filtro derivativo}
    {Kp_master_500_Kp_slave_500_Ki_slave_\ki_Kd_slave_\kd_Kn_slave_\kn}
    {K_{p_{\small{mestre}}}=500 \quad K_{p_{\small{escravo}}}=500%
    \quad K_{i_{\small{escravo}}}=\ki \quad K_{d_{\small{escravo}}}=\kd%
    \quad K_{n_{\small{escravo}}}=\kn }
    }
\end{figure}


\newpage

\subsubsection{Controlador Escravo PID filtro windup}

\begin{figure}[h]
  \foreach \ki/\kd/\kaw in {100/100/1,300/100/2,2000/100/3}{
      \labMasterSlaveSubFigure{P}{PID filtro windup}
    {Kp_master_500_Kp_slave_500_Ki_slave_\ki_Kd_slave_\kd_Kaw_slave_\kaw}
    {K_{p_{\small{mestre}}}=500 \quad K_{p_{\small{escravo}}}=500%
    \quad K_{i_{\small{escravo}}}=\ki \quad K_{d_{\small{escravo}}}=\kd%
    \quad K_{aw_{\small{escravo}}}=\kaw }
    }
\end{figure}