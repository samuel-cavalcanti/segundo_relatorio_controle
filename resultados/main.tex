%
%
Primeiramente, houveram testes de controle de primeira ordem, para entrada constante de valor 15 e senoidal,
$f(t)= 2sen(t)+10$

Depois, testes de controle de segunda ordem, para entrada constante de valor 15 e senoidal, de amplitude 2, bias 10 e frequência de 0.2 rad/s. Foram implementados intertravamentos para evitar transbordamento dos tanques ou succionar água (tensão negativa) quando o nível do primeiro tanque estiver muito baixo ou desligar (\emph{shutdown}) o sistema caso haja uma ocorrência anormal.

Testes de primeira e segunda ordem (sistemas 1 e 2) foram agrupados e discutidos em conjunto.

Por fim foram realizados testes com o controlador em cascata (mestre-escravo) e foi discutido como este tipo de configuração influencia no desempenho do sistema, em comparação ao sistemas de tipo 2 discutido anteriormente.

As imagens foram agrupadas por tipo de controlador e respostas para entrada constante e senoidal, dispostas lado a lado, para sistema tipo 1 (primeira ordem) e sistema tipo 2 (segunda ordem), como descrito na legenda de cada grupo.
\newpage
%
%
% 
% 
% 
\subsection{Controlador P}\hspace{4ex}

\begin{figure}[h]
    \foreach \kpSystemOne/\kpSystemTwo in {5/10,10/20,20/100}{
        \labABSubFigure{1}{P}{kp_\kpSystemOne}{K_p=\kpSystemOne}%1-1
        \labABSubFigure{2}{P}{kp_\kpSystemTwo}{K_p=\kpSystemTwo}%2-1
  }
\end{figure}

\hspace{4ex}
Testes de controle de primeira ordem, para entrada constante de valor 15 e senoidal, de amplitude 2 e bias 10.

Pode-se observar que, no sistema de tipo 1, com o aumento do ganho proporcional houve um
aumento na velocidade do sistema até a resposta em regime permanente e
uma diminuição do erro estacionário, ou seja,
a diferença entre a resposta desejada e a resposta em regime permanente. No senoidal ocorreu o mesmo, aumentando $K_p$ o nível do tanque 1 passa a seguir a referência e o nível do segundo passa a orbitar o bias, sem nunca estabilizar por causa da baixa frequência.

Observou-se que também que, no sistema de tipo 2,
o aumento de $K_p$ aumentou a oscilação do sistema mas diminuiu o 
erro de regime, não zerando-o completamente, como esperado. Sendo assim,
o sistema é do tipo subamortecido (0 < $\epsilon$ < 1), e considerando
que $K_p$ = $\omega_n^2$, a frequência aumenta com o ganho. Houve aumento
no tempo de acomodação e haveria também aumento significativo no
\emph{overshoot}, como se pode observar pela resposta do controlador
que é quase uma onda quadrada em $K_p$ = 100.
Não ocorreu devido a saturação do sistema. Com entrada senoidal, o aumento do $K_p$ resultou na resposta do segundo tanque seguir exatamente a referência e oscilações no nível do primeiro tanque.


\newpage

\subsection{Controlador PI}\hspace{4ex}
\begin{figure}[h]
    \foreach \kiSystemOne/\kiSystemTwo in {0.5/0.01,5/0.5,10/0.005}{
        \labABSubFigure{1}{PI}{kp_20_ki_\kiSystemOne}{K_p=20 \quad K_i=\kiSystemOne}%1-1
        \labABSubFigure{2}{PI}{kp_20_ki_\kiSystemTwo}{K_p=20 \quad K_i=\kiSystemTwo}%2-1
    }

\end{figure}

\hspace{4ex}


No sistema tipo 1, com a adição do controlador integral o sistema passa a ter comportamento de segunda ordem, subamortecido no exemplo, zerando o erro de regime observado anteriormente. No entanto, há bastante oscilação e o sistema é mais lento, chegando ao regime permanente em cerca de 3 vezes o tempo de sua contraparte apenas proporcional. Também há o custo do \emph{overshoot} de aproximadamente 36\% quando $K_i$ = 10, como observado. Para entrada senoidal o \emph{overshoot} também cresce com o aumento de $K_i$.

No sistema tipo 2, o erro de regime é automaticamente zerado, no entanto o intertravamento precisa agir mesmo para valores diminutos do ganho integral $K_i$ em 0.5, para que o tanque 1 não transborde (passe do nível 29). O efeito cumulativo da parcela integral do controlador, quando tem seu ganho aumentado, gera uma resposta mais oscilatória e com maior \emph{overshoot}, até que instabiliza o sistema em $K_i$ muito altos. Nesta situação o sistema continua a oscilar sem nunca chegar ao regime permanente em tempo plausível. Essa tendencia foi observada no experimento. Para entrada senoidal, o aumento de $K_i$ elevou o \emph{overshoot} do nível do segundo tanque e a oscilação do nível do primeiro, como esperado.


\newpage

\subsection{Controlador PD}\hspace{4ex}

\begin{figure}[h]
    \foreach \kdSystemOne/\kdSystemTwo in {1/1,5/15,15/25}{
        \labABSubFigure{1}{PD}{kp_20_kd_\kdSystemOne}{K_p=20 \quad K_d=\kdSystemOne}%1-1
        \labABSubFigure{2}{PD}{kp_20_kd_\kdSystemTwo}{K_p=20 \quad K_d=\kdSystemTwo}%2-1
    }
\end{figure}

\hspace{4ex}
Escrever sobre o Controlador PD

\newpage

\subsection{Controlador PID}\hspace{4ex}
\begin{figure}[h]
    \foreach \kiSystemOne/\kdSystemOne/\kiSystemTwo/\kdSystemTwo in {0.5/22/1/250,1/22/1/500,2/22/1/1000}{
        \labABSubFigure{1}{PID}{kp_20_ki_\kiSystemOne_kd_\kdSystemOne}
        {K_p=20 \quad K_i=\kiSystemOne \quad K_d=\kdSystemOne}%1-1
%
        \labABSubFigure{2}{PID}{kp_20_ki_\kiSystemTwo_kd_\kdSystemTwo}
        {K_p=20 \quad K_i=\kiSystemTwo \quad K_d=\kdSystemTwo}%2-1
    }
\end{figure}

\hspace{4ex}

Nesse contexto, o controlador proporcional ajusta a variável de controle proporcionalmente ao erro, o integral ajusta proporcional ao erro acumulado e a derivativa proporcional a velocidade de variação do erro. Permite um maior controle do comportamento do sistema, em relação aos anteriores. Observou-se um aumento no \emph{overshoot} ao se aumentar $K_i$ e uma maior oscilação com aumento de $K_p$.

Como observado no controlador PI, instabilização ocorre para maiores valores de $K_i$, onde o sistema nunca alcança regime permanente. O mesmo efeito ocorre ao se alterar $K_p$ mas não na mesma proporção. Alterar $K_d$ afeta notavelmente o tempo de acomodação e a sensibilidade do controlador ao efeito do ruído, como visto no do tipo PD. Para valores menores de $K_d$ (notadamente $K_d$ = 100) a frequência diminui de tal maneira a aumentar o tempo de acomodação. Para entrada senoidal, foi possível ajustar $K_i$ e $K_d$ para mitigar os efeitos de oscilação no nível do primeiro tanque ao mesmo tempo que se reduz a velocidade de regime transitório. 



\newpage

\subsection{Controlador PID com filtro na ação derivativa}\hspace{4ex}

\def \PIDDerivativeFilter{PID filtro derivativo}

\begin{figure}[h]
    \foreach \knSystemOne/\knSystemTwo in {0.5/30,50/50,100/100}{
        \labABSubFigure{1}{\PIDDerivativeFilter}
        {kp_20_ki_1_kd_1_kn_\knSystemOne}
        {K_p=20 \quad K_i=1 \quad K_d=1 \quad K_n=\knSystemOne}%1-1
%
        \labABSubFigure{2}{\PIDDerivativeFilter}
        {kp_20_ki_1_kd_1_kn_\knSystemTwo}
        {K_p=20 \quad K_i=1 \quad K_d=1 \quad K_n=\knSystemTwo}%2-1
  }
\end{figure}

\hspace{4ex}
Escrever sobre o Controlador PID com filtro na ação derivativa

\newpage

\subsection{Controlador PID com filtro anti-reset-windup}

\def \PIDFilter{PID filtro windup}

\begin{figure}[h]
\labABSubFigure{1}{\PIDFilter}{kp_0_5_ki_0_5_kd_0_05_kaw_0_5}{K_p=0.5 \quad K_i=0.5 \quad K_d=0.05 \quad K_{aw}=0.5}%1-1
\labABSubFigure{2}{\PIDFilter}{kp_20_ki_1_kd_1_kaw_1}{K_p=20 \quad K_i=1 \quad K_d=1 \quad K_{aw}=1}%2-1
\labABSubFigure{1}{\PIDFilter}{kp_0_5_ki_0_5_kd_0_05_kaw_1}{K_p=0.5 \quad K_i=0.5 \quad K_d=0.05 \quad K_{aw}=1}%1-2
\labABSubFigure{2}{\PIDFilter}{kp_20_ki_1_kd_20_kaw_1}{K_p=20 \quad K_i=1 \quad K_d=20 \quad K_{aw}=1}%2-2
\labABSubFigure{1}{\PIDFilter}{kp_0_5_ki_1_kd_0_05_kaw_0_5}{K_p=0.5 \quad K_i=1 \quad K_d=0.05 \quad K_{aw}=0.5}%1-3
\labABSubFigure{2}{\PIDFilter}{kp_20_ki_5_kd_1_kaw_1}{K_p=20 \quad K_i=5 \quad K_d=1 \quad K_{aw}=1}%2-3
\end{figure}

\hspace{4ex}
Quando o sistema está sujeito a um erro constante diferente de zero, o controlador integral tende a acumular seu efeito e requisitar comandos cada vez maiores ao atuador, mesmo estando saturado. Quando o erro eventualmente decresce e se torna negativo, há um tempo até que a saída integral “retorne” ao seu estado funcional, é o que se chama de tempo de \emph{windup}. O filtro \emph{anti-windup} contrabalanceia esse efeito somando ou subtraindo um valor proporcional ao ganho $K_aw$ quando há saturação (saída do controlador e entrada da planta são diferentes).

No primeiro sistema, o filtro \emph{anti-windup} atenua o efeito da parcela integral do controlador. Com $K_aw$ = 1 há uma leve redução do \emph{overshoot} e do tempo de acomodação. A frequência natural não foi modificada ao aumentar $K_aw$, apenas a mesma redução citada anteriormente, \emph{overshoot} por exemplo cai de 12\% para 9\%  aumentando o ganho do filtro de $K_aw$ 0.5 para 1. O \emph{overshoot} do tanque 2 é completamente removido para valores menores de $K_i$. Para entrada senoidal, o aumento de $K_aw$ 0.001 para 0.01 deslocou o bias dos níveis em -0.5, aproximadamente.

No segundo sistema, mesmo com $K_i$ = 1 a oscilação é controlada até que se aumente seu valor para algo mais considerável (no caso 5), onde há \emph{overshoot}. Aumentar $K_d$ tem efeito de redução da amplitude das oscilações, como esperado. Com entrada senoidal 


% 
% 
% 
% 
\subsection{Sistema de Controle Mestre-Escravo: Mestre P}

\subsubsection{Controlador Escravo P}
\begin{figure}[h]
\foreach \kp in {40,300,500}{
 \labMasterSlaveSubFigure{P}{P}{kp_master_\kp_kp_slave_\kp}
    {K_{p_{\textrm{mestre}}}=\kp \quad K_{p_{\textrm{escravo}}}=\kp }%
}
\end{figure}

Podemos Observar que quando maior o valor de $K_p$ mais próximo o nível do tanque se aproxíma da referência, também foi observado que a partir de 300
aumentar o seu valor deixa de surgir efeito. Resultados semelhetes encontrado com um controlador P, mas  diferente do controlador P, foi observado que é possível
controlar a variação do tanque 1, onde para valores baixos de  $K_{p_{\textrm{Escravo}}}$ a variação do nível do tanque 1 é baixa analogamente, valores altos aumenta a variação
do nível 1.

\newpage

\subsubsection{Controlador Escravo PD}

\begin{figure}[h]
\foreach \kdSlave in {100,1000,1000}{
    \labMasterSlaveSubFigure{P}{PD}{Kp_master_500_Kp_slave_500_Kd_slave_\kdSlave}
    {K_{p_{\textrm{mestre}}}=500 \quad K_{p_{\textrm{escravo}}}=500%
    \quad K_{d_{\textrm{escravo}}}=\kdSlave}%
    }
\end{figure}

Escrever sobre o Controlador PD

\newpage

\subsubsection{Controlador Escravo PI}

\begin{figure}[h]
\foreach \kiSlave in {3,6,12}{
    \labMasterSlaveSubFigure{P}{PI}{Kp_master_500_Kp_slave_500_Ki_slave_\kiSlave}
    {K_{p_{\textrm{mestre}}}=500 \quad K_{p_{\textrm{escravo}}}=500%
    \quad K_{i_{\textrm{escravo}}}=\kiSlave}%
    }
\end{figure}

Escrever sobre o Controlador PI

\newpage


\subsubsection{Controlador Escravo PID}

\begin{figure}[h]
    \foreach \ki/\kd in {0.5/5000,1/300,3/300}{
    \labMasterSlaveSubFigure{P}{PID}
    {Kp_master_500_Kp_slave_500_Ki_slave_\ki_Kd_slave_\kd}
    {K_{p_{\textrm{mestre}}}=500 \quad K_{p_{\textrm{escravo}}}=500%
    \quad K_{i_{\textrm{escravo}}}=\ki \quad K_{d_{\textrm{escravo}}}=\kd}%
    }
\end{figure}

Uma vez colocado o PID no Controlador escravo foi possível observar que apesar das constantes de controle influênciarem
fortemente no nível do tanque 1, é possível observar que essas influência alcança o nível 2, uma vez observado uma subida
sútil no gráfico a.

\newpage

\subsubsection{Controlador Escravo PID filtro derivativo}

\begin{figure}[h]
  \foreach \ki/\kd/\kn in {1/100/2,1/100/15,7/100/15}{
      \labMasterSlaveSubFigure{P}{PID filtro derivativo}
    {Kp_master_500_Kp_slave_500_Ki_slave_\ki_Kd_slave_\kd_Kn_slave_\kn}
    {K_{p_{\textrm{mestre}}}=500 \quad K_{p_{\textrm{escravo}}}=500%
    \quad K_{i_{\textrm{escravo}}}=\ki \quad K_{d_{\textrm{escravo}}}=\kd%
    \quad K_{n_{\textrm{escravo}}}=\kn }
    }
\end{figure}

Escrever sobre o Controlador PID com filtro na ação derivativa

\newpage

\subsubsection{Controlador Escravo PID filtro windup}

\begin{figure}[h]
  \foreach \ki/\kd/\kaw in {100/100/1,300/100/2,2000/100/3}{
      \labMasterSlaveSubFigure{P}{PID filtro windup}
    {Kp_master_500_Kp_slave_500_Ki_slave_\ki_Kd_slave_\kd_Kaw_slave_\kaw}
    {K_{p_{\textrm{mestre}}}=500 \quad K_{p_{\textrm{escravo}}}=500%
    \quad K_{i_{\textrm{escravo}}}=\ki \quad K_{d_{\textrm{escravo}}}=\kd%
    \quad K_{aw_{\textrm{escravo}}}=\kaw }
    }
\end{figure}
%
O controlador Escravo PID com o Windup, permite diminuir o overshooting, a variação do nível 1 e o tempo de decida
do nível do tanque 1 sem precisar aumentar as constantes derivativas.
% 
% 
\newpage
%
% 
\subsection{Sistema de Controle Mestre-Escravo: Mestre PD}
%
\def \currentMaster{Mestre PD}
\def \currentSlave{escravo P}
\def \masterSlaveCaption{ \currentMaster - \currentSlave }
%
\def \controllerPEvaluation{resultados/roteiro_c/mestre_PD/Controlador_P}
\def \controllerPDEvaluation{resultados/roteiro_c/mestre_PD/Controlador_PD}
\def \controllerPIEvaluation{resultados/roteiro_c/mestre_PD/Controlador_PI}
\def \controllerPIDEvaluation{resultados/roteiro_c/mestre_PD/Controlador_PID}
\def \controllerPIDDevvaluation{resultados/roteiro_c/mestre_PD/Controlador_PID_filtro_dev}
\def \controllerPIDWindupEvaluation{resultados/roteiro_c/mestre_PD/Controlador_PID_filtro_windup}
%
%
\subsubsection{Controlador Escravo P}
\begin{figure}[h]
\foreach \kd in {500,1000,10000}{ %kp_master_500_kd_master_500_kp_slave_500
 \labMasterSlaveSubFigure{PD}{P}{kp_master_500_kd_master_\kd_kp_slave_500}
    {K_{p_{\textrm{mestre}}}=500 \quad K_{d_{\textrm{mestre}}}=\kd \quad K_{p_{\textrm{escravo}}}=500 }%
}
\controllerCaption{1}{\masterSlaveCaption}
\end{figure}

\input{\controllerPEvaluation}

\newpage
%
\def \currentSlave{escravo PD}
%
\subsubsection{Controlador Escravo PD}

\begin{figure}[h]
\foreach \kdMaster/\kdSlave in {1/1,1/100,5/100}{
    \labMasterSlaveSubFigure{PD}{PD}
    {Kp_master_500_Kd_master_\kdMaster_Kp_slave_500_Kd_slave_\kdSlave}
    {K_{p_{\textrm{mestre}}}=500 \quad K_{d_{\textrm{mestre}}}=\kdMaster
    \quad K_{p_{\textrm{escravo}}}=500 \quad K_{d_{\textrm{escravo}}}=\kdSlave}%
    }
    \controllerCaption{1}{\masterSlaveCaption}
\end{figure}

\input{\controllerPDEvaluation}

\newpage
%
\def \currentSlave{escravo PI}
%

\subsubsection{Controlador Escravo PI}

\begin{figure}[h]
\foreach \kdMaster/\kiSlave in {1/10,1/50,10000/10}{
    \labMasterSlaveSubFigure{PD}{PI}
    {Kp_master_500_Kd_master_\kdMaster_Kp_slave_500_Ki_slave_\kiSlave}
    {K_{p_{\textrm{mestre}}}=500 \quad K_{d_{\textrm{mestre}}}=\kdMaster \quad K_{p_{\textrm{escravo}}}=500 \quad K_{i_{\textrm{escravo}}}=\kiSlave}%
    }
    \controllerCaption{1}{\masterSlaveCaption}
\end{figure}

\input{\controllerPIEvaluation}

\newpage
%
\def \currentSlave{escravo PID}
%


\subsubsection{Controlador Escravo PID}

\begin{figure}[h]
\foreach  \kdMaster/\kiSlave/\kdSlave in {1/10/500,1/20/2000,2/20/2000}{
\labMasterSlaveSubFigure{PD}{PID}
{Kp_master_500_Kd_master_\kdMaster_kp_slave_500_ki_slave_\kiSlave_kd_slave_\kdSlave}
{K_{p_{\textrm{mestre}}}=500 \quad K_{d_{\textrm{mestre}}}=\kdMaster \quad K_{p_{\textrm{escravo}}}=500 \quad K_{i_{\textrm{escravo}}}=\kiSlave \quad K_{d_{\textrm{escravo}}}=\kdSlave}
}
\controllerCaption{1}{\masterSlaveCaption}
\end{figure}

\input{\controllerPIDEvaluation}

\newpage
%
\def \currentSlave{escravo PID filtro derivativo}
%
\subsubsection{Controlador Escravo PID filtro derivativo}

\begin{figure}[h]
  \foreach \kdMaster/\ki/\kd/\kn in {1/20/2000/1.5,2/20/2000/1.5,2/20/2500/0.75}{
      \labMasterSlaveSubFigure{PD}{PID filtro derivativo}
    {Kp_master_500_Kd_master_\kdMaster_Kp_slave_500_Ki_slave_\ki_Kd_slave_\kd_Kn_slave_\kn}
    {K_{p_{\textrm{mestre}}}=500 \quad K_{d_{\textrm{master}}}=\kdMaster%
    K_{p_{\textrm{escravo}}}=500 \quad K_{i_{\textrm{escravo}}}=\ki%
    \quad K_{d_{\textrm{escravo}}}=\kd \quad K_{n_{\textrm{escravo}}}=\kn}
    }
    \controllerCaption{1}{\masterSlaveCaption}
\end{figure}
\input{\controllerPIDDevvaluation}

\newpage
%
\def \currentSlave{escravo PID filtro windup}
%
\subsubsection{Controlador Escravo PID filtro windup}

\begin{figure}[h]
  \foreach \kdMaster/\ki/\kd/\kaw in {1/10/2/0.005,2.5/50/2.5/0.01,3/50/3/0.15}{
      \labMasterSlaveSubFigure{PD}{PID filtro windup}
    {Kp_master_500_Kd_master_\kdMaster_Kp_slave_500_Ki_slave_\ki_Kd_slave_\kd_Kaw_slave_\kaw}
    {K_{p_{\textrm{mestre}}}=500 \quad K_{d_{\textrm{mestre}}}=\kdMaster \quad %
     K_{p_{\textrm{escravo}}}=500 \quad K_{i_{\textrm{escravo}}}=\ki \quad  %
    K_{d_{\textrm{escravo}}}=\kd \quad K_{aw_{\textrm{escravo}}}=\kaw }
    }
    \controllerCaption{1}{\masterSlaveCaption}
\end{figure}

\input{\controllerPIDWindupEvaluation}

\newpage
%
% 
\subsection{Sistema de Controle Mestre-Escravo: Mestre PI
}
%
\def \currentMaster{PI}
\def \currentSlave{escravo P}
\def \masterSlaveCaption{Mestre \currentMaster - \currentSlave }
\def \rootDir{mestre_PI}
%
\def \controllerPEvaluation{resultados/roteiro_c/\rootDir/Controlador_P}
\def \controllerPDEvaluation{resultados/roteiro_c/\rootDir/Controlador_PD}
\def \controllerPIEvaluation{resultados/roteiro_c/\rootDir/Controlador_PI}
\def \controllerPIDEvaluation{resultados/roteiro_c/\rootDir/Controlador_PID}
\def \controllerPIDDevvaluation{resultados/roteiro_c/\rootDir/Controlador_PID_filtro_dev}
\def \controllerPIDWindupEvaluation{resultados/roteiro_c/\rootDir/Controlador_PID_filtro_windup}
%
%
%
%
\subsubsection{Controlador Escravo P}
\begin{figure}[h]
    \labMasterSlaveSubFigure{PI}
{P}
{Kp_master_500_Ki_master_2_Kp_slave_500}
{K_{p_{\textrm{mestre}}}=500 \quad K_{i_{\textrm{mestre}}}=2 \quad K_{p_{\textrm{escravo}}}=500}
\labMasterSlaveSubFigure{PI}
{P}
{Kp_master_500_Ki_master_20_Kp_slave_500}
{K_{p_{\textrm{mestre}}}=500 \quad K_{i_{\textrm{mestre}}}=20 \quad K_{p_{\textrm{escravo}}}=500}
\labMasterSlaveSubFigure{PI}
{P}
{Kp_master_500_Ki_master_50_Kp_slave_500}
{K_{p_{\textrm{mestre}}}=500 \quad K_{i_{\textrm{mestre}}}=50 \quad K_{p_{\textrm{escravo}}}=500}

    \controllerCaption{1}{\masterSlaveCaption}
\end{figure}

\input{\controllerPEvaluation}

\newpage
%
\def \currentSlave{escravo PD}
%
\subsubsection{Controlador Escravo PD}

\begin{figure}[h]
    \labMasterSlaveSubFigure{PI}
{PD}
{Kp_master_500_Ki_master_0_5_Kp_slave_500_Kd_slave_10000}
{K_{p_{\textrm{mestre}}}=500 \quad K_{i_{\textrm{mestre}}}=0.5 \quad K_{p_{\textrm{escravo}}}=500 \quad K_{d_{\textrm{escravo}}}=10000}
\labMasterSlaveSubFigure{PI}
{PD}
{Kp_master_500_Ki_master_0_5_Kp_slave_500_Kd_slave_1000}
{K_{p_{\textrm{mestre}}}=500 \quad K_{i_{\textrm{mestre}}}=0.5 \quad K_{p_{\textrm{escravo}}}=500 \quad K_{d_{\textrm{escravo}}}=1000}
\labMasterSlaveSubFigure{PI}
{PD}
{Kp_master_500_Ki_master_2_Kp_slave_500_Kd_slave_300}
{K_{p_{\textrm{mestre}}}=500 \quad K_{i_{\textrm{mestre}}}=2 \quad K_{p_{\textrm{escravo}}}=500 \quad K_{d_{\textrm{escravo}}}=300}

    \controllerCaption{1}{\masterSlaveCaption}
\end{figure}

\input{\controllerPDEvaluation}

\newpage
%
\def \currentSlave{escravo PI}
%

\subsubsection{Controlador Escravo PI}

\begin{figure}[h]
    \labMasterSlaveSubFigure{PI}
{PI}
{Kp_master_500_Ki_master_0_5_Kp_slave_500_Ki_slave_0_5}
{K_{p_{\textrm{mestre}}}=500 \quad K_{i_{\textrm{mestre}}}=0.5 \quad K_{p_{\textrm{escravo}}}=500 \quad K_{i_{\textrm{escravo}}}=0.5}
\labMasterSlaveSubFigure{PI}
{PI}
{Kp_master_500_Ki_master_0_5_Kp_slave_500_Ki_slave_1}
{K_{p_{\textrm{mestre}}}=500 \quad K_{i_{\textrm{mestre}}}=0.5 \quad K_{p_{\textrm{escravo}}}=500 \quad K_{i_{\textrm{escravo}}}=1}
\labMasterSlaveSubFigure{PI}
{PI}
{Kp_master_500_Ki_master_0_5_Kp_slave_500_Ki_slave_5}
{K_{p_{\textrm{mestre}}}=500 \quad K_{i_{\textrm{mestre}}}=0.5 \quad K_{p_{\textrm{escravo}}}=500 \quad K_{i_{\textrm{escravo}}}=5}

    \controllerCaption{1}{\masterSlaveCaption}
\end{figure}

\input{\controllerPIEvaluation}

\newpage
%
\def \currentSlave{escravo PID}
%


\subsubsection{Controlador Escravo PID}

\begin{figure}[h]
    \labMasterSlaveSubFigure{PI}
{PID}
{Kp_master_500_Ki_master_0_5_Kp_slave_500_Ki_slave_0_5_Kd_slave_10000}
{K_{p_{\textrm{mestre}}}=500 \quad K_{i_{\textrm{mestre}}}=0.5 \quad K_{p_{\textrm{escravo}}}=500 \quad K_{i_{\textrm{escravo}}}=0.5 \quad K_{d_{\textrm{escravo}}}=10000}
\labMasterSlaveSubFigure{PI}
{PID}
{Kp_master_500_Ki_master_0_5_Kp_slave_500_Ki_slave_0_5_Kd_slave_1000}
{K_{p_{\textrm{mestre}}}=500 \quad K_{i_{\textrm{mestre}}}=0.5 \quad K_{p_{\textrm{escravo}}}=500 \quad K_{i_{\textrm{escravo}}}=0.5 \quad K_{d_{\textrm{escravo}}}=1000}
\labMasterSlaveSubFigure{PI}
{PID}
{Kp_master_500_Ki_master_0_5_Kp_slave_500_Ki_slave_0_5_Kd_slave_500}
{K_{p_{\textrm{mestre}}}=500 \quad K_{i_{\textrm{mestre}}}=0.5 \quad K_{p_{\textrm{escravo}}}=500 \quad K_{i_{\textrm{escravo}}}=0.5 \quad K_{d_{\textrm{escravo}}}=500}

    \controllerCaption{1}{\masterSlaveCaption}
\end{figure}

\input{\controllerPIDEvaluation}

\newpage
%
\def \currentSlave{escravo PID filtro derivativo}
%
\subsubsection{Controlador Escravo PID filtro derivativo}

\begin{figure}[h]
    \labMasterSlaveSubFigure{PI}
{PID filtro derivativo}
{Kp_master_500_Ki_master_2_Kp_slave_500_Ki_slave_0_5_Kd_slave_10000_Kn_slave_1}
{K_{p_{\textrm{mestre}}}=500 \quad K_{i_{\textrm{mestre}}}=2 \quad K_{p_{\textrm{escravo}}}=500 \quad K_{i_{\textrm{escravo}}}=0.5 \quad K_{d_{\textrm{escravo}}}=10000 \quad K_{n_{\textrm{escravo}}}=1}
\labMasterSlaveSubFigure{PI}
{PID filtro derivativo}
{Kp_master_500_Ki_master_2_Kp_slave_500_Ki_slave_0_5_Kd_slave_10000_Kn_slave_2}
{K_{p_{\textrm{mestre}}}=500 \quad K_{i_{\textrm{mestre}}}=2 \quad K_{p_{\textrm{escravo}}}=500 \quad K_{i_{\textrm{escravo}}}=0.5 \quad K_{d_{\textrm{escravo}}}=10000 \quad K_{n_{\textrm{escravo}}}=2}
\labMasterSlaveSubFigure{PI}
{PID filtro derivativo}
{Kp_master_500_Ki_master_30_Kp_slave_500_Ki_slave_0_5_Kd_slave_10000_Kn_slave_4_5}
{K_{p_{\textrm{mestre}}}=500 \quad K_{i_{\textrm{mestre}}}=30 \quad K_{p_{\textrm{escravo}}}=500 \quad K_{i_{\textrm{escravo}}}=0.5 \quad K_{d_{\textrm{escravo}}}=10000 \quad K_{n_{\textrm{escravo}}}=4.5}

    \controllerCaption{1}{\masterSlaveCaption}
\end{figure}
\input{\controllerPIDDevvaluation}

\newpage
%
\def \currentSlave{escravo PID filtro windup}
%
\subsubsection{Controlador Escravo PID filtro windup}

\begin{figure}[h]
    \labMasterSlaveSubFigure{PI}
{PID filtro windup}
{Kp_master_500_Ki_master_200_Kp_slave_500_Ki_slave_0_5_Kd_slave_45_Kaw_slave_2}
{K_{p_{\textrm{mestre}}}=500 \quad K_{i_{\textrm{mestre}}}=200 \quad K_{p_{\textrm{escravo}}}=500 \quad K_{i_{\textrm{escravo}}}=0.5 \quad K_{d_{\textrm{escravo}}}=45 \quad K_{aw_{\textrm{escravo}}}=2}
\labMasterSlaveSubFigure{PI}
{PID filtro windup}
{Kp_master_500_Ki_master_300_Kp_slave_500_Ki_slave_0_5_Kd_slave_45_Kaw_slave_2}
{K_{p_{\textrm{mestre}}}=500 \quad K_{i_{\textrm{mestre}}}=300 \quad K_{p_{\textrm{escravo}}}=500 \quad K_{i_{\textrm{escravo}}}=0.5 \quad K_{d_{\textrm{escravo}}}=45 \quad K_{aw_{\textrm{escravo}}}=2}
\labMasterSlaveSubFigure{PI}
{PID filtro windup}
{Kp_master_500_Ki_master_300_Kp_slave_500_Ki_slave_0_5_Kd_slave_45_Kaw_slave_5}
{K_{p_{\textrm{mestre}}}=500 \quad K_{i_{\textrm{mestre}}}=300 \quad K_{p_{\textrm{escravo}}}=500 \quad K_{i_{\textrm{escravo}}}=0.5 \quad K_{d_{\textrm{escravo}}}=45 \quad K_{aw_{\textrm{escravo}}}=5}

    \controllerCaption{1}{\masterSlaveCaption}
\end{figure}

\input{\controllerPIDWindupEvaluation}

\newpage
%
% 
\subsection{Sistema de Controle Mestre-Escravo: Mestre PID
}
%
\def \currentMaster{PID}
\def \currentSlave{escravo P}
\def \masterSlaveCaption{Mestre \currentMaster - \currentSlave }
\def \rootDir{mestre_PID}
%
\def \controllerPEvaluation{resultados/roteiro_c/\rootDir/Controlador_P}
\def \controllerPDEvaluation{resultados/roteiro_c/\rootDir/Controlador_PD}
\def \controllerPIEvaluation{resultados/roteiro_c/\rootDir/Controlador_PI}
\def \controllerPIDEvaluation{resultados/roteiro_c/\rootDir/Controlador_PID}
\def \controllerPIDDevvaluation{resultados/roteiro_c/\rootDir/Controlador_PID_filtro_dev}
\def \controllerPIDWindupEvaluation{resultados/roteiro_c/\rootDir/Controlador_PID_filtro_windup}
%
%
%
%
\subsubsection{Controlador Escravo P}
\begin{figure}[h]
    \labMasterSlaveSubFigure{PID}
{P}
{Kp_master_500_Ki_master_1_Kd_master_1_Kp_slave_500}
{K_{p_{\textrm{mestre}}}=500 \quad K_{i_{\textrm{mestre}}}=2 \quad K_{d_{\textrm{mestre}}}=1 \quad K_{p_{\textrm{escravo}}}=500}
\labMasterSlaveSubFigure{PID}
{P}
{Kp_master_500_Ki_master_1_Kd_master_500_Kp_slave_500}
{K_{p_{\textrm{mestre}}}=500 \quad K_{i_{\textrm{mestre}}}=1 \quad K_{d_{\textrm{mestre}}}=500 \quad K_{p_{\textrm{escravo}}}=500}
\labMasterSlaveSubFigure{PID}
{P}
{Kp_master_500_Ki_master_30_Kd_master_500_Kp_slave_500}
{K_{p_{\textrm{mestre}}}=500 \quad K_{i_{\textrm{mestre}}}=30 \quad K_{d_{\textrm{mestre}}}=500 \quad K_{p_{\textrm{escravo}}}=500}

    \controllerCaption{1}{\masterSlaveCaption}
\end{figure}

\input{\controllerPEvaluation}

\newpage
%
\def \currentSlave{escravo PD}
%
\subsubsection{Controlador Escravo PD}

\begin{figure}[h]
    \labMasterSlaveSubFigure{PID}
{PD}
{Kp_master_500_Ki_master_30_Kd_master_500_Kp_slave_500_Kd_slave_2}
{K_{p_{\textrm{mestre}}}=500 \quad K_{i_{\textrm{mestre}}}=30 \quad K_{d_{\textrm{mestre}}}=500 \quad K_{p_{\textrm{escravo}}}=500 \quad K_{d_{\textrm{escravo}}}=2}
\labMasterSlaveSubFigure{PID}
{PD}
{Kp_master_500_Ki_master_30_Kd_master_500_Kp_slave_500_Kd_slave_5}
{K_{p_{\textrm{mestre}}}=500 \quad K_{i_{\textrm{mestre}}}=30\quad K_{d_{\textrm{mestre}}}=500 \quad K_{p_{\textrm{escravo}}}=500 \quad K_{d_{\textrm{escravo}}}=5}
\labMasterSlaveSubFigure{PID}
{PD}
{Kp_master_500_Ki_master_30_Kd_master_500_Kp_slave_500_Kd_slave_10}
{K_{p_{\textrm{mestre}}}=500 \quad K_{i_{\textrm{mestre}}}=30 \quad K_{d_{\textrm{mestre}}}=500 \quad K_{p_{\textrm{escravo}}}=500 \quad K_{d_{\textrm{escravo}}}=10}

    \controllerCaption{1}{\masterSlaveCaption}
\end{figure}

\input{\controllerPDEvaluation}

\newpage
%
\def \currentSlave{escravo PI}
%

\subsubsection{Controlador Escravo PI}

\begin{figure}[h]
    \labMasterSlaveSubFigure{PID}
{PI}
{Kp_master_500_Ki_master_30_Kd_master_500_Kp_slave_500_Ki_slave_5}
{K_{p_{\textrm{mestre}}}=500 \quad K_{i_{\textrm{mestre}}}=30 \quad K_{d_{\textrm{mestre}}}=500 \quad K_{p_{\textrm{escravo}}}=500 \quad K_{i_{\textrm{escravo}}}=5}
\labMasterSlaveSubFigure{PID}
{PI}
{Kp_master_500_Ki_master_30_Kd_master_500_Kp_slave_500_Ki_slave_10}
{K_{p_{\textrm{mestre}}}=500 \quad K_{i_{\textrm{mestre}}}=30 \quad K_{d_{\textrm{mestre}}}=500 \quad K_{p_{\textrm{escravo}}}=500 \quad K_{i_{\textrm{escravo}}}=10}
\labMasterSlaveSubFigure{PID}
{PI}
{Kp_master_500_Ki_master_30_Kd_master_10000_Kp_slave_500_Ki_slave_0_5}
{K_{p_{\textrm{mestre}}}=500 \quad K_{i_{\textrm{mestre}}}=30 \quad K_{d_{\textrm{mestre}}}=10000 \quad K_{p_{\textrm{escravo}}}=500 \quad K_{i_{\textrm{escravo}}}=0.5}

    \controllerCaption{1}{\masterSlaveCaption}
\end{figure}

\input{\controllerPIEvaluation}

\newpage
%
\def \currentSlave{escravo PID}
%


\subsubsection{Controlador Escravo PID}

\begin{figure}[h]
    \labMasterSlaveSubFigure{PID}
{PID}
{Kp_master_500_Ki_master_30_Kd_master_500_Kp_slave_500_Ki_slave_0_5_Kd_slave_2}
{K_{p_{\textrm{mestre}}}=500 \quad K_{i_{\textrm{mestre}}}=30 \quad K_{d_{\textrm{mestre}}}=500 \quad K_{p_{\textrm{escravo}}}=500 \quad K_{i_{\textrm{escravo}}}=0.5 \quad K_{d_{\textrm{escravo}}}=2}
\labMasterSlaveSubFigure{PID}
{PID}
{Kp_master_500_Ki_master_30_Kd_master_500_Kp_slave_500_Ki_slave_0_5_Kd_slave_5}
{K_{p_{\textrm{mestre}}}=500 \quad K_{i_{\textrm{mestre}}}=30 \quad K_{d_{\textrm{mestre}}}=500 \quad K_{p_{\textrm{escravo}}}=500 \quad K_{i_{\textrm{escravo}}}=0.5 \quad K_{d_{\textrm{escravo}}}=5}
\labMasterSlaveSubFigure{PID}
{PID}
{Kp_master_500_Ki_master_30_Kd_master_10000_Kp_slave_500_Ki_slave_0_5_Kd_slave_1}
{K_{p_{\textrm{mestre}}}=500 \quad K_{i_{\textrm{mestre}}}=30 \quad K_{d_{\textrm{mestre}}}=10000 \quad K_{p_{\textrm{escravo}}}=500 \quad K_{i_{\textrm{escravo}}}=0.5 \quad K_{d_{\textrm{escravo}}}=1}

    \controllerCaption{1}{\masterSlaveCaption}
\end{figure}

\input{\controllerPIDEvaluation}

\newpage
%
\def \currentSlave{escravo PID filtro derivativo}
%
\subsubsection{Controlador Escravo PID filtro derivativo}

\begin{figure}[h]
    \labMasterSlaveSubFigure{PID}
{PID filtro derivativo}
{Kp_master_500_Ki_master_30_Kd_master_500_Kp_slave_500_Ki_slave_0_5_Kd_slave_5_Kn_slave_1}
{K_{p_{\textrm{mestre}}}=500 \quad K_{i_{\textrm{mestre}}}=30 \quad K_{d_{\textrm{mestre}}}=500 \quad K_{p_{\textrm{escravo}}}=500 \quad K_{i_{\textrm{escravo}}}=0.5 \quad K_{d_{\textrm{escravo}}}=10000 \quad K_{n_{\textrm{escravo}}}=1}
\labMasterSlaveSubFigure{PID}
{PID filtro derivativo}
{Kp_master_500_Ki_master_30_Kd_master_500_Kp_slave_500_Ki_slave_0_5_Kd_slave_5_Kn_slave_2}
{K_{p_{\textrm{mestre}}}=500 \quad K_{i_{\textrm{mestre}}}=30 \quad K_{d_{\textrm{mestre}}}=500 \quad K_{p_{\textrm{escravo}}}=500 \quad K_{i_{\textrm{escravo}}}=0.5 \quad K_{d_{\textrm{escravo}}}=10000 \quad K_{n_{\textrm{escravo}}}=2}
\labMasterSlaveSubFigure{PID}
{PID filtro derivativo}
{Kp_master_500_Ki_master_30_Kd_master_500_Kp_slave_500_Ki_slave_0_5_Kd_slave_5_Kn_slave_3}
{K_{p_{\textrm{mestre}}}=500 \quad K_{i_{\textrm{mestre}}}=30 \quad K_{d_{\textrm{mestre}}}=500 \quad K_{p_{\textrm{escravo}}}=500 \quad K_{i_{\textrm{escravo}}}=0.5 \quad K_{d_{\textrm{escravo}}}=10000 \quad K_{n_{\textrm{escravo}}}=3}

    \controllerCaption{1}{\masterSlaveCaption}
\end{figure}
\input{\controllerPIDDevvaluation}

\newpage
%
\def \currentSlave{escravo PID filtro windup}
%
\subsubsection{Controlador Escravo PID filtro windup}

\begin{figure}[h]
    \labMasterSlaveSubFigure{PID}
{PID filtro windup}
{Kp_master_500_Ki_master_5_Kd_master_500_Kp_slave_500_Ki_slave_5_Kd_slave_1_Kaw_slave_0_005}
{K_{p_{\textrm{mestre}}}=500 \quad K_{i_{\textrm{mestre}}}=5 \quad K_{d_{\textrm{mestre}}}=500 \quad K_{p_{\textrm{escravo}}}=500 \quad K_{i_{\textrm{escravo}}}=5 \quad K_{d_{\textrm{escravo}}}=1 \quad K_{aw_{\textrm{escravo}}}=0.005}
\labMasterSlaveSubFigure{PID}
{PID filtro windup}
{Kp_master_500_Ki_master_30_Kd_master_500_Kp_slave_500_Ki_slave_0_5_Kd_slave_5_Kaw_slave_0_005}
{K_{p_{\textrm{mestre}}}=500 \quad K_{i_{\textrm{mestre}}}=30 \quad K_{d_{\textrm{mestre}}}=500 \quad K_{p_{\textrm{escravo}}}=500 \quad K_{i_{\textrm{escravo}}}=0.5 \quad K_{d_{\textrm{escravo}}}=5 \quad K_{aw_{\textrm{escravo}}}=0.005}
\labMasterSlaveSubFigure{PID}
{PID filtro windup}
{Kp_master_500_Ki_master_30_Kd_master_500_Kp_slave_500_Ki_slave_30_Kd_slave_1_Kaw_slave_0_005}
{K_{p_{\textrm{mestre}}}=500 \quad K_{i_{\textrm{mestre}}}=30 \quad K_{d_{\textrm{mestre}}}=500 \quad K_{p_{\textrm{escravo}}}=500 \quad K_{i_{\textrm{escravo}}}=30 \quad K_{d_{\textrm{escravo}}}=1 \quad K_{aw_{\textrm{escravo}}}=0.005}

    \controllerCaption{1}{\masterSlaveCaption}
\end{figure}

\input{\controllerPIDWindupEvaluation}

\newpage
%
% 
\subsection{Sistema de Controle Mestre-Escravo: Mestre PID filtro derivativo
}
%
\def \currentMaster{PID filtro derivativo}
\def \currentSlave{escravo P}
\def \masterSlaveCaption{Mestre \currentMaster - \currentSlave }
\def \rootDir{mestre_PID_filtro_derivativo}
%
\def \controllerPEvaluation{resultados/roteiro_c/\rootDir/Controlador_P}
\def \controllerPDEvaluation{resultados/roteiro_c/\rootDir/Controlador_PD}
\def \controllerPIEvaluation{resultados/roteiro_c/\rootDir/Controlador_PI}
\def \controllerPIDEvaluation{resultados/roteiro_c/\rootDir/Controlador_PID}
\def \controllerPIDDevvaluation{resultados/roteiro_c/\rootDir/Controlador_PID_filtro_dev}
\def \controllerPIDWindupEvaluation{resultados/roteiro_c/\rootDir/Controlador_PID_filtro_windup}
%
%
%
%
\subsubsection{Controlador Escravo P}
\begin{figure}[h]
    \labMasterSlaveSubFigure{PID filtro derivativo}
{P}
{Kp_master_500_Ki_master_1_Kd_master_1_Kn_master_1_Kp_slave_500}
{K_{p_{\textrm{mestre}}}=500 \quad K_{i_{\textrm{mestre}}}=1 \quad K_{d_{\textrm{mestre}}}=1 \quad K_{n_{\textrm{mestre}}}=1 \quad K_{p_{\textrm{escravo}}}=500}
\labMasterSlaveSubFigure{PID filtro derivativo}
{P}
{Kp_master_500_Ki_master_7_Kd_master_1_Kn_master_200_Kp_slave_500}
{K_{p_{\textrm{mestre}}}=500 \quad K_{i_{\textrm{mestre}}}=7 \quad K_{d_{\textrm{mestre}}}=1 \quad K_{n_{\textrm{mestre}}}=200 \quad K_{p_{\textrm{escravo}}}=500}
\labMasterSlaveSubFigure{PID filtro derivativo}
{P}
{Kp_master_500_Ki_master_7_Kd_master_1_Kn_master_500_Kp_slave_500}
{K_{p_{\textrm{mestre}}}=500 \quad K_{i_{\textrm{mestre}}}=7 \quad K_{d_{\textrm{mestre}}}=1 \quad K_{n_{\textrm{mestre}}}=500 \quad K_{p_{\textrm{escravo}}}=500}

    \controllerCaption{1}{\masterSlaveCaption}
\end{figure}

\input{\controllerPEvaluation}

\newpage
%
\def \currentSlave{escravo PD}
%
\subsubsection{Controlador Escravo PD}

\begin{figure}[h]
    \labMasterSlaveSubFigure{PID filtro derivativo}
{PD}
{Kp_master_500_Ki_master_100_Kd_master_1_Kn_master_500_Kp_slave_500_Kd_slave_2}
{K_{p_{\textrm{mestre}}}=500 \quad K_{i_{\textrm{mestre}}}=100 \quad K_{d_{\textrm{mestre}}}=1 \quad K_{n_{\textrm{mestre}}}=500 \quad K_{p_{\textrm{escravo}}}=500 \quad K_{d_{\textrm{escravo}}}=2}
\labMasterSlaveSubFigure{PID filtro derivativo}
{PD}
{Kp_master_500_Ki_master_100_Kd_master_1_Kn_master_500_Kp_slave_500_Kd_slave_3}
{K_{p_{\textrm{mestre}}}=500 \quad K_{i_{\textrm{mestre}}}=100\quad K_{d_{\textrm{mestre}}}=1 \quad K_{n_{\textrm{mestre}}}=500 \quad K_{p_{\textrm{escravo}}}=500 \quad K_{d_{\textrm{escravo}}}=3}
\labMasterSlaveSubFigure{PID filtro derivativo}
{PD}
{Kp_master_500_Ki_master_100_Kd_master_1_Kn_master_500_Kp_slave_500_Kd_slave_5}
{K_{p_{\textrm{mestre}}}=500 \quad K_{i_{\textrm{mestre}}}=100 \quad K_{d_{\textrm{mestre}}}=1 \quad K_{n_{\textrm{mestre}}}=500 \quad K_{p_{\textrm{escravo}}}=500 \quad K_{d_{\textrm{escravo}}}=5}

    \controllerCaption{1}{\masterSlaveCaption}
\end{figure}

\input{\controllerPDEvaluation}

\newpage
%
\def \currentSlave{escravo PI}
%

\subsubsection{Controlador Escravo PI}

\begin{figure}[h]
    \labMasterSlaveSubFigure{PID filtro derivativo}
{PI}
{Kp_master_500_Ki_master_1_Kd_master_1_Kn_master_750_Kp_slave_500_Ki_slave_14}
{K_{p_{\textrm{mestre}}}=500 \quad K_{i_{\textrm{mestre}}}=1 \quad K_{d_{\textrm{mestre}}}=1 \quad K_{n_{\textrm{mestre}}}=750 \quad K_{p_{\textrm{escravo}}}=500 \quad K_{i_{\textrm{escravo}}}=14}
\labMasterSlaveSubFigure{PID filtro derivativo}
{PI}
{Kp_master_500_Ki_master_3_Kd_master_1_Kn_master_500_Kp_slave_500_Ki_slave_3}
{K_{p_{\textrm{mestre}}}=500 \quad K_{i_{\textrm{mestre}}}=3 \quad K_{d_{\textrm{mestre}}}=1 \quad K_{n_{\textrm{mestre}}}=500 \quad K_{p_{\textrm{escravo}}}=500 \quad K_{i_{\textrm{escravo}}}=3}
\labMasterSlaveSubFigure{PID filtro derivativo}
{PI}
{Kp_master_500_Ki_master_7_Kd_master_1_Kn_master_500_Kp_slave_500_Ki_slave_3}
{K_{p_{\textrm{mestre}}}=500 \quad K_{i_{\textrm{mestre}}}=7 \quad K_{d_{\textrm{mestre}}}=1 \quad K_{n_{\textrm{mestre}}}=500 \quad K_{p_{\textrm{escravo}}}=500 \quad K_{i_{\textrm{escravo}}}=3}

    \controllerCaption{1}{\masterSlaveCaption}
\end{figure}

\input{\controllerPIEvaluation}

\newpage
%
\def \currentSlave{escravo PID}
%


\subsubsection{Controlador Escravo PID}

\begin{figure}[h]
    \labMasterSlaveSubFigure{PID filtro derivativo}
{PID}
{Kp_master_2000_Ki_master_4_Kd_master_1_2_Kn_master_400_Kp_slave_2000_Ki_slave_4_Kd_slave_1}
{K_{p_{\textrm{mestre}}}=2000 \quad K_{i_{\textrm{mestre}}}=4 \quad K_{d_{\textrm{mestre}}}=1.2 \quad K_{n_{\textrm{mestre}}}=400 \quad K_{p_{\textrm{escravo}}}=2000 \quad K_{i_{\textrm{escravo}}}=4 \quad K_{d_{\textrm{escravo}}}=1}
\labMasterSlaveSubFigure{PID filtro derivativo}
{PID}
{Kp_master_2000_Ki_master_4_Kd_master_1_Kn_master_200_Kp_slave_2000_Ki_slave_4_Kd_slave_1}
{K_{p_{\textrm{mestre}}}=2000 \quad K_{i_{\textrm{mestre}}}=4 \quad K_{d_{\textrm{mestre}}}=1  \quad K_{n_{\textrm{mestre}}}=200\quad K_{p_{\textrm{escravo}}}=2000 \quad K_{i_{\textrm{escravo}}}=4 \quad K_{d_{\textrm{escravo}}}=1}
\labMasterSlaveSubFigure{PID filtro derivativo}
{PID}
{Kp_master_2000_Ki_master_4_Kd_master_1_Kn_master_400_Kp_slave_2000_Ki_slave_4_Kd_slave_1}
{K_{p_{\textrm{mestre}}}=2000 \quad K_{i_{\textrm{mestre}}}=4 \quad K_{d_{\textrm{mestre}}}=1 \quad K_{n_{\textrm{mestre}}}=400 \quad K_{p_{\textrm{escravo}}}=2000 \quad K_{i_{\textrm{escravo}}}=4 \quad K_{d_{\textrm{escravo}}}=1}

    \controllerCaption{1}{\masterSlaveCaption}
\end{figure}

\input{\controllerPIDEvaluation}

\newpage
%
\def \currentSlave{escravo PID filtro derivativo}
%
\subsubsection{Controlador Escravo PID filtro derivativo}

\begin{figure}[h]
    \labMasterSlaveSubFigure{PID filtro derivativo}
{PID filtro derivativo}
{Kp_master_2000_Ki_master_6_Kd_master_1_2_Kn_master_400_Kp_slave_2000_Ki_slave_4_Kd_slave_0_001_Kn_slave_2}
{K_{p_{\textrm{mestre}}}=2000 \quad K_{i_{\textrm{mestre}}}=6 \quad K_{d_{\textrm{mestre}}}=1.2  \quad K_{n_{\textrm{mestre}}}=400  \quad K_{p_{\textrm{escravo}}}=2000 \quad K_{i_{\textrm{escravo}}}=4 \quad K_{d_{\textrm{escravo}}}=0.001 \quad K_{n_{\textrm{escravo}}}=2}
\labMasterSlaveSubFigure{PID filtro derivativo}
{PID filtro derivativo}
{Kp_master_2000_Ki_master_6_Kd_master_1_2_Kn_master_400_Kp_slave_2000_Ki_slave_6_Kd_slave_0_01_Kn_slave_700}
{K_{p_{\textrm{mestre}}}=2000 \quad K_{i_{\textrm{mestre}}}=6 \quad K_{d_{\textrm{mestre}}}=1.2 \quad K_{n_{\textrm{mestre}}}=400  \quad K_{p_{\textrm{escravo}}}=2000 \quad K_{i_{\textrm{escravo}}}=6 \quad K_{d_{\textrm{escravo}}}=0.01 \quad K_{n_{\textrm{escravo}}}=700}
\labMasterSlaveSubFigure{PID filtro derivativo}
{PID filtro derivativo}
{Kp_master_2000_Ki_master_6_Kd_master_2_5_Kn_master_400_Kp_slave_2000_Ki_slave_6_Kd_slave_0_01_Kn_slave_700}
{K_{p_{\textrm{mestre}}}=2000 \quad K_{i_{\textrm{mestre}}}=6 \quad K_{d_{\textrm{mestre}}}=2.5 \quad K_{n_{\textrm{mestre}}}=400  \quad K_{p_{\textrm{escravo}}}=2000 \quad K_{i_{\textrm{escravo}}}=6 \quad K_{d_{\textrm{escravo}}}=0.01 \quad K_{n_{\textrm{escravo}}}=700}

    \controllerCaption{1}{\masterSlaveCaption}
\end{figure}
\input{\controllerPIDDevvaluation}

\newpage
%
\def \currentSlave{escravo PID filtro windup}
%
\subsubsection{Controlador Escravo PID filtro windup}

\begin{figure}[h]
    \labMasterSlaveSubFigure{PID filtro derivativo}
{PID filtro windup}
{Kp_master_200_Ki_master_1_Kd_master_0_01_Kp_slave_200_Ki_slave_2_Kd_slave_100_Kaw_slave_0_002}
{K_{p_{\textrm{mestre}}}=200 \quad K_{i_{\textrm{mestre}}}=1 \quad K_{d_{\textrm{mestre}}}=0.01 \quad K_{p_{\textrm{escravo}}}=200 \quad K_{i_{\textrm{escravo}}}=2 \quad K_{d_{\textrm{escravo}}}=100 \quad K_{aw_{\textrm{escravo}}}=0.002}
\labMasterSlaveSubFigure{PID filtro derivativo}
{PID filtro windup}
{Kp_master_200_Ki_master_1_Kd_master_0_05_Kp_slave_200_Ki_slave_1_Kd_slave_1_Kaw_slave_0_001}
{K_{p_{\textrm{mestre}}}=200 \quad K_{i_{\textrm{mestre}}}=1 \quad K_{d_{\textrm{mestre}}}=0.05 \quad K_{p_{\textrm{escravo}}}=200 \quad K_{i_{\textrm{escravo}}}=1 \quad K_{d_{\textrm{escravo}}}=1 \quad K_{aw_{\textrm{escravo}}}=0.001}
\labMasterSlaveSubFigure{PID filtro derivativo}
{PID filtro windup}
{Kp_master_200_Ki_master_1_Kd_master_0_05_Kp_slave_200_Ki_slave_1_Kd_slave_700_Kaw_slave_0_0001}
{K_{p_{\textrm{mestre}}}=200 \quad K_{i_{\textrm{mestre}}}=1 \quad K_{d_{\textrm{mestre}}}=0.05 \quad K_{p_{\textrm{escravo}}}=200 \quad K_{i_{\textrm{escravo}}}=1 \quad K_{d_{\textrm{escravo}}}=700 \quad K_{aw_{\textrm{escravo}}}=0.0001}

    \controllerCaption{1}{\masterSlaveCaption}
\end{figure}

\input{\controllerPIDWindupEvaluation}
