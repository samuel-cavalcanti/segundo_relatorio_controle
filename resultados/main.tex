% newcommand define novos comandos, que podem passar a ser usados da
% mesma forma que os comandos LaTeX de base.
%
% % Implicação em fórmulas
% \newcommand{\implica}{\quad\Rightarrow\quad} %Meio de linha
% \newcommand{\implicafim}{\quad\Rightarrow}   %Fim de linha
% \newcommand{\tende}{\rightarrow}
% \newcommand{\BibTeX}{\textsc{B\hspace{-0.1em}i\hspace{-0.1em}b\hspace{-0.3em}}\TeX}
%
% % Fração com parentesis
% \newcommand{\pfrac}[2]{\left(\frac{#1}{#2}\right)}
%
% % Transformada de Laplace e transformada Z
% %\newcommand{\lapl}{\makebox[0cm][l]{\hspace{0.1em}\raisebox{0.25ex}{-}}\mathcal{L}}
% \newcommand{\lapl}{\pounds}
% \newcommand{\transfz}{\mathcal{Z}}
%
% % Não aparecer o número na primeira página dos capítulos
% \newcommand{\mychapter}[1]{\chapter{#1}\thispagestyle{empty}}

% % Os capítulos sem número
% \newcommand{\mychapterast}[1]{\chapter*{#1}\thispagestyle{empty}
% \chaptermark{#1}
% \afterpage{\markboth{\uppercase{#1}}{\rightmark}}
% \markboth{\uppercase{#1}}{}
% }
%
% % Seções sem número
% \newcommand{\mysectionast}[1]{\section*{#1}
% \addcontentsline{toc}{section}{#1}
% \markright{\uppercase{#1}}
% }
%
% % No tabularx, as celulas devem ser centradas verticalmente
% \renewcommand{\tabularxcolumn}[1]{m{#1}}
%
% % Células centralizadas horizontalmente no tabularx
% \newcolumntype{C}{>{\centering\arraybackslash}X}
%
% %% Abrevia figuras e tabelas
% %\def\figurename{Fig.}
% %\def\tablename{Tab.}
%
%
\newcommand{\labABSubFigure}[4]{
% o tipo do sistema 1 ou 2
% tipo do controlador P
% final do arquivo -> #3
% titulo
\ifnum#1=1
    \def \roteiro {roteiro a}
    \def \sistema {sistema tipo 1 }
\fi
\ifnum#1=2
    \def \roteiro {roteiro b}
    \def \sistema {sistema tipo 2 }
\fi
%
\def \controlador{controlador #2}
\def \ganho{#3.pdf}
\def \caminhoImagem{images/\roteiro/\controlador/\ganho}
%
%
\begin{subfigure}[b]{.5\textwidth}
  \centering
  \includegraphics[width=.8\linewidth]{\caminhoImagem}
  \caption{\sistema $#4$}
\end{subfigure}%
}

\newcommand{\labMasterSlaveSubFigure}[4]{
% o tipo do sistema 1 ou 2
% tipo do controlador P
% final do arquivo -> #3
% titulo
\def \controladorMestre{controlador Mestre #1}
\def \controladorEscravo{Controlador escravo #2}
\def \ganho{#3.pdf}
\def \caminhoImagem{images/roteiro c/\controladorMestre/\controladorEscravo/\ganho}
%
%
\begin{subfigure}[b]{.5\textwidth}
  \includegraphics[width=.8\linewidth]{\caminhoImagem}
  \caption{\controladorEscravo \\\hspace{.5\textwidth} \tiny $#4$ }
\end{subfigure}%
}

\newcommand{\controllerCaption}[2]{
% o nome do  controlador #1
% o tipo da entrada #2
%
\ifnum#1=1
    \def \sinalReferencia{a entrada constante}
\fi
\ifnum#1=2
    \def \sinalReferencia{a entrada senoidal}
\fi
%$
  \caption{resposta do Controlador #2 \sinalReferencia}
}

Primeiramente, houveram testes de controle de primeira ordem, para entrada constante de valor 15 e senoidal, de amplitude 2 e bias 10.

Depois, testes de controle de segunda ordem, para entrada constante de valor 15. Foram implementados intertravamentos para evitar transbordamento dos tanques ou succionar água (tensão negativa) quando o nível do primeiro tanque estiver muito baixo ou desligar (\emph{shutdown}) o sistema caso haja uma ocorrência anormal.

Testes de primeira e segunda ordem (sistemas 1 e 2) foram agrupados e discutidos em conjunto.

Por fim foram realizados testes com o controlador em cascata (mestre-escravo) e foi discutido como este tipo de configuração influencia no desempenho do sistema, em comparação ao sistemas de tipo 2 discutido anteriormente.


% 
% 
% 
\subsection{Controlador P}\hspace{4ex}

\begin{figure}[h]
    \foreach \kpSystemOne/\kpSystemTwo in {5/10,10/20,20/100}{
        \labABSubFigure{1}{P}{kp_\kpSystemOne}{K_p=\kpSystemOne}%1-1
        \labABSubFigure{2}{P}{kp_\kpSystemTwo}{K_p=\kpSystemTwo}%2-1
  }
\end{figure}

\hspace{4ex}
Testes de controle de primeira ordem, para entrada constante de valor 15 e senoidal, de amplitude 2 e bias 2.

Pode-se observar que, no sistema de tipo 1, com o aumento do ganho proporcional houve um
aumento na velocidade do sistema até a resposta em regime permanente e
uma diminuição do erro estacionário, ou seja,
a diferença entre a resposta desejada e a resposta em regime permanente.

Observou-se que também que, no sistema de tipo 2,
o aumento de $K_p$ aumentou a oscilação do sistema mas diminuiu o 
erro de regime, não zerando-o completamente, como esperado. Sendo assim,
o sistema é do tipo subamortecido (0 < $\epsilon$ < 1), e considerando
que $K_p$ = $\omega_n^2$, a frequência aumenta com o ganho. Houve aumento
no tempo de acomodação e haveria também aumento significativo no
\emph{overshoot}, como se pode observar pela resposta do controlador
que é quase uma onda quadrada em $K_p$ = 100.
Não ocorreu devido a saturação do sistema.


\newpage

\subsection{Controlador PI}\hspace{4ex}
\begin{figure}[h]
    \foreach \kiSystemOne/\kiSystemTwo in {0.5/0.01,5/0.5,10/0.005}{
        \labABSubFigure{1}{PI}{kp_20_ki_\kiSystemOne}{K_p=20 \quad K_i=\kiSystemOne}%1-1
        \labABSubFigure{2}{PI}{kp_20_ki_\kiSystemTwo}{K_p=20 \quad K_i=\kiSystemTwo}%2-1
    }

\end{figure}

\hspace{4ex}


No sistema tipo 1, com a adição do controlador integral o sistema passa a ter comportamento de segunda ordem, subamortecido no exemplo, zerando o erro de regime observado anteriormente. No entanto, há bastante oscilação e o sistema é mais lento, chegando ao regime permanente em cerca de 3 vezes o tempo de sua contraparte apenas proporcional. Também há o custo do \emph{overshoot} de aproximadamente 36\% quando $K_i$ = 10, como observado.

No sistema tipo 2, o erro de regime é automaticamente zerado, no entanto o intertravamento precisa agir mesmo para valores diminutos do ganho integral $K_i$ em 0.5, para que o tanque 1 não transborde (passe do nível 29). O efeito cumulativo da parcela integral do controlador, quando tem seu ganho aumentado, gera uma resposta mais oscilatória e com maior \emph{overshoot}, até que instabiliza o sistema em $K_i$ muito altos. Nesta situação o sistema continua a oscilar sem nunca chegar ao regime permanente em tempo plausível. Essa tendencia foi observada no experimento.


\newpage

\subsection{Controlador PD}\hspace{4ex}

\begin{figure}[h]
    \foreach \kdSystemOne/\kdSystemTwo in {1/1,5/15,15/25}{
        \labABSubFigure{1}{PD}{kp_20_kd_\kdSystemOne}{K_p=20 \quad K_d=\kdSystemOne}%1-1
        \labABSubFigure{2}{PD}{kp_20_kd_\kdSystemTwo}{K_p=20 \quad K_d=\kdSystemTwo}%2-1
    }
\end{figure}

\hspace{4ex}

A ação derivativa nesse contexto tem função de antecipar a ação de controle, ao levar em conta a “inércia” do erro, gerando maior estabilidade relativa e velocidade de resposta transitória. Com isso se observou o aumento do erro relativo a medida em que se aumentada o ganho diferencial $K_d$. Também é atestado a amplificação do ruído, já que o mesmo é formado por bruscas variações em curtos períodos de tempo. Esse ruído amplificado, no entanto, é saturado em -4/+4. 

Para entrada senoidal no sistema 1 o bias de referência foi afetado: a medida em que se aumentava $K_d$ a média permanente do sinal diminuía em relação a referencia. No caso, de 10 para 9 em $K_d$ = 100.

Já no sistema 2, ao aumentar o ganho diferencial $K_d$ o \emph{overshoot} foi atenuado em aproximadamente 10\% de $K_d$ = 1 para $K_d$ = 25, enquanto a sensibilidade do controlador as alterações advindas do ruído foram bastante acentuadas. Além disso, nesse contexto, a velocidade da resposta transitória foi praticamente dobrada (tempo de acomodação cai pela metade). Na entrada senoidal o aumento do $K_d$ reduziu oscilações no nível do tanque 1 e \emph{overshoot} inicial, além disso, fez o nível do 2 seguir melhor a referência.


\newpage

\subsection{Controlador PID}\hspace{4ex}
\begin{figure}[h]
    \foreach \kiSystemOne/\kdSystemOne/\kiSystemTwo/\kdSystemTwo in {0.5/22/1/250,1/22/1/500,2/22/1/1000}{
        \labABSubFigure{1}{PID}{kp_20_ki_\kiSystemOne_kd_\kdSystemOne}
        {K_p=20 \quad K_i=\kiSystemOne \quad K_d=\kdSystemOne}%1-1
%
        \labABSubFigure{2}{PID}{kp_20_ki_\kiSystemTwo_kd_\kdSystemTwo}
        {K_p=20 \quad K_i=\kiSystemTwo \quad K_d=\kdSystemTwo}%2-1
    }
\end{figure}

\hspace{4ex}

Nesse contexto, o controlador proporcional ajusta a variável de controle proporcionalmente ao erro, o integral ajusta proporcional ao erro acumulado e a derivativa proporcional a velocidade de variação do erro. Permite um maior controle do comportamento do sistema, em relação aos anteriores. Observou-se um aumento no \emph{overshoot} ao se aumentar $K_i$ e uma maior oscilação com aumento de $K_p$.

Como observado no controlador PI, instabilização ocorre para maiores valores de $K_i$, onde o sistema nunca alcança regime permanente. O mesmo efeito ocorre ao se alterar $K_p$ mas não na mesma proporção. Alterar $K_d$ afeta notavelmente o tempo de acomodação e a sensibilidade do controlador ao efeito do ruído, como visto no do tipo PD. Para valores menores de $K_d$ (notadamente $K_d$ = 100) a frequência diminui de tal maneira a aumentar o tempo de acomodação. Para entrada senoidal, foi possível ajustar $K_i$ e $K_d$ para mitigar os efeitos de oscilação no nível do primeiro tanque ao mesmo tempo que se reduz a velocidade de regime transitório. 



\newpage

\subsection{Controlador PID com filtro na ação derivativa}\hspace{4ex}

\def \PIDDerivativeFilter{PID filtro derivativo}

\begin{figure}[h]
    \foreach \knSystemOne/\knSystemTwo in {0.5/30,50/50,100/100}{
        \labABSubFigure{1}{\PIDDerivativeFilter}
        {kp_20_ki_1_kd_1_kn_\knSystemOne}
        {K_p=20 \quad K_i=1 \quad K_d=1 \quad K_n=\knSystemOne}%1-1
%
        \labABSubFigure{2}{\PIDDerivativeFilter}
        {kp_20_ki_1_kd_1_kn_\knSystemTwo}
        {K_p=20 \quad K_i=1 \quad K_d=1 \quad K_n=\knSystemTwo}%2-1
  }
\end{figure}

\hspace{4ex}
O controlador derivativo está suscetível a amplificar ruído de baixa amplitude e alta frequência, podendo comprometer o sistema. É possível criar um laço de realimentação com uma integral no retorno e ganho N, que age como um filtro passa-baixas, atenuando o efeito do ruído, como se pode observar ao comparar a amplitude de oscilações com e sem filtro. Para entrada senoidal no sistema tipo 1, o nível do primeiro tanque passou a seguir levemente o nível do segundo e diminuição  em 9\% do \emph{overshoot} em ao ajustar $K_n$ de 2 para 5.

Em ambos sistemas observou-se eliminação do \emph{overshoot} e aumento no tempo de acomodação para valores altos de $K_n$ (notadamente $K_n$ $\geq$ 100). Além disso a onda passou a "deformar" já que nessa situação se coloca muito peso na resposta influenciada por ruído aleatório.

No sistema do tipo 2 e entrada senoidal, o aumento de $K_n$ desloca a fase do nível do segundo tanque e "satura" o nível do primeiro em 4, como é observado em (f).

\newpage

\subsection{Controlador PID com filtro anti-reset-windup}

\def \PIDFilter{PID filtro windup}

\begin{figure}[h]
\labABSubFigure{1}{\PIDFilter}{kp_0_5_ki_0_5_kd_0_05_kaw_0_5}{K_p=0.5 \quad K_i=0.5 \quad K_d=0.05 \quad K_{aw}=0.5}%1-1
\labABSubFigure{2}{\PIDFilter}{kp_20_ki_1_kd_1_kaw_1}{K_p=20 \quad K_i=1 \quad K_d=1 \quad K_{aw}=1}%2-1
\labABSubFigure{1}{\PIDFilter}{kp_0_5_ki_0_5_kd_0_05_kaw_1}{K_p=0.5 \quad K_i=0.5 \quad K_d=0.05 \quad K_{aw}=1}%1-2
\labABSubFigure{2}{\PIDFilter}{kp_20_ki_1_kd_20_kaw_1}{K_p=20 \quad K_i=1 \quad K_d=20 \quad K_{aw}=1}%2-2
\labABSubFigure{1}{\PIDFilter}{kp_0_5_ki_1_kd_0_05_kaw_0_5}{K_p=0.5 \quad K_i=1 \quad K_d=0.05 \quad K_{aw}=0.5}%1-3
\labABSubFigure{2}{\PIDFilter}{kp_20_ki_5_kd_1_kaw_1}{K_p=20 \quad K_i=5 \quad K_d=1 \quad K_{aw}=1}%2-3
\end{figure}

\hspace{4ex}
Quando o sistema está sujeito a um erro constante diferente de zero, o controlador integral tende a acumular seu efeito e requisitar comandos cada vez maiores ao atuador, mesmo estando saturado. Quando o erro eventualmente decresce e se torna negativo, há um tempo até que a saída integral “retorne” ao seu estado funcional, é o que se chama de tempo de \emph{windup}. O filtro \emph{anti-windup} contrabalanceia esse efeito somando ou subtraindo um valor proporcional ao ganho $K_aw$ quando há saturação (saída do controlador e entrada da planta são diferentes).

O filtro \emph{anti-windup} atenua o efeito da parcela integral do controlador. Com $K_aw$ = 0.05 há uma leve redução do \emph{overshoot} e do tempo de acomodação. A frequência natural não foi modificada ao aumentar $K_aw$, apenas a mesma redução citada anteriormente, \emph{overshoot} por exemplo cai de 50\% para 35\% e tempo de acomodação (aproximadamente 5\%) de 180 ms para 130 ms aumentando o ganho do filtro de 0.05 para 0.5. O \emph{overshoot} do tanque 2 é completamente removido para valores menores de $K_i$.


% 
% 
% 
% 
\subsection{Sistema de Controle Mestre-Escravo: Mestre P}

\def \currentMaster{Mestre P}
\def \currentSlave{escravo P}
\def \masterSlaveCaption{ \currentMaster - \currentSlave }

\subsubsection{Controlador Escravo P}
\begin{figure}[h]
\foreach \kp in {40,300,500}{
 \labMasterSlaveSubFigure{P}{P}{kp_master_\kp_kp_slave_\kp}
    {K_{p_{\textrm{mestre}}}=\kp \quad K_{p_{\textrm{escravo}}}=\kp }%
}
\controllerCaption{1}{\masterSlaveCaption}
\end{figure}

Podemos Observar que quando maior o valor de $K_p$ mais próximo o nível do tanque se aproxíma da referência, também foi observado que a partir de 300
aumentar o seu valor deixa de surgir efeito. Resultados semelhetes encontrado com um controlador P, mas  diferente do controlador P, foi observado que é possível
controlar a variação do tanque 1, onde para valores baixos de  $K_{p_{\textrm{Escravo}}}$ a variação do nível do tanque 1 é baixa analogamente, valores altos aumenta a variação
do nível 1.

\newpage
%
\def \currentSlave{escravo PD}
%
\subsubsection{Controlador Escravo PD}

\begin{figure}[h]
\foreach \kdSlave in {100,1000,1000}{
    \labMasterSlaveSubFigure{P}{PD}{Kp_master_500_Kp_slave_500_Kd_slave_\kdSlave}
    {K_{p_{\textrm{mestre}}}=500 \quad K_{p_{\textrm{escravo}}}=500%
    \quad K_{d_{\textrm{escravo}}}=\kdSlave}%
    }
    \controllerCaption{1}{\masterSlaveCaption}
\end{figure}

Como também observado no Escravo P, só que dessa vez fica mais nítido que o controlador escravo possuí influência no nível do tanque 1. Observamos que
aumentar o  $K_{d_{\textrm{Escravo}}}$ diminui o overshooting do tanque 1 e acelera a decida para o nível de referência.

\newpage

%
\def \currentSlave{escravo PI}
%

\subsubsection{Controlador Escravo PI}

\begin{figure}[h]
\foreach \kiSlave in {3,6,12}{
    \labMasterSlaveSubFigure{P}{PI}{Kp_master_500_Kp_slave_500_Ki_slave_\kiSlave}
    {K_{p_{\textrm{mestre}}}=500 \quad K_{p_{\textrm{escravo}}}=500%
    \quad K_{i_{\textrm{escravo}}}=\kiSlave}%
    }
    \controllerCaption{1}{\masterSlaveCaption}
\end{figure}

analogamente ao Controlador escravo PD, o PI aumenta overshooting e a variação do nível do tanque 1.

\newpage
%
\def \currentSlave{escravo PID}
%

\subsubsection{Controlador Escravo PID}

\begin{figure}[h]
    \foreach \ki/\kd in {0.5/5000,1/300,3/300}{
    \labMasterSlaveSubFigure{P}{PID}
    {Kp_master_500_Kp_slave_500_Ki_slave_\ki_Kd_slave_\kd}
    {K_{p_{\textrm{mestre}}}=500 \quad K_{p_{\textrm{escravo}}}=500%
    \quad K_{i_{\textrm{escravo}}}=\ki \quad K_{d_{\textrm{escravo}}}=\kd}%
    }
    \controllerCaption{1}{\masterSlaveCaption}
\end{figure}

Uma vez colocado o PID no Controlador escravo foi possível observar que apesar das constantes de controle influênciarem
fortemente no nível do tanque 1, é possível observar que essas influência alcança o nível 2, uma vez observado uma subida
sútil no gráfico a.

\newpage
%
\def \currentSlave{escravo PID filtro derivativo}
%
\subsubsection{Controlador Escravo PID filtro derivativo}

\begin{figure}[h]
  \foreach \ki/\kd/\kn in {1/100/2,1/100/15,7/100/15}{
      \labMasterSlaveSubFigure{P}{PID filtro derivativo}
    {Kp_master_500_Kp_slave_500_Ki_slave_\ki_Kd_slave_\kd_Kn_slave_\kn}
    {K_{p_{\textrm{mestre}}}=500 \quad K_{p_{\textrm{escravo}}}=500%
    \quad K_{i_{\textrm{escravo}}}=\ki \quad K_{d_{\textrm{escravo}}}=\kd%
    \quad K_{n_{\textrm{escravo}}}=\kn }
    }
    \controllerCaption{1}{\masterSlaveCaption}
\end{figure}

Com o Filtro derivativo no Controlador PID escravo fica mais evidente que é alterações no valores das constantes de
controle do Controlador escravo interfere no tempo de regime e overshooting do tanque 2,Também é observado que
a constante $K_{n_{\textrm{escravo}}}$ ajuda o  $K_{d_{\textrm{escravo}}}$ a ajudara suavizar o overshooting e
e aumentar o tempo de transição.

\newpage
%
\def \currentSlave{escravo PID filtro windup}
%
\subsubsection{Controlador Escravo PID filtro windup}

\begin{figure}[h]
  \foreach \ki/\kd/\kaw in {100/100/1,300/100/2,2000/100/3}{
      \labMasterSlaveSubFigure{P}{PID filtro windup}
    {Kp_master_500_Kp_slave_500_Ki_slave_\ki_Kd_slave_\kd_Kaw_slave_\kaw}
    {K_{p_{\textrm{mestre}}}=500 \quad K_{p_{\textrm{escravo}}}=500%
    \quad K_{i_{\textrm{escravo}}}=\ki \quad K_{d_{\textrm{escravo}}}=\kd%
    \quad K_{aw_{\textrm{escravo}}}=\kaw }
    }
    \controllerCaption{1}{\masterSlaveCaption}
\end{figure}
%
O controlador Escravo PID com o Windup, permite diminuir o overshooting, a variação do nível 1 e o tempo de decida
do nível do tanque 1 sem precisar aumentar as constantes derivativas.
% 
% 
\newpage
%
% 
\subsection{Sistema de Controle Mestre-Escravo: Mestre PD}

\subsubsection{Controlador Escravo P}
\begin{figure}[h]
\foreach \kd in {500,1000,10000}{ %kp_master_500_kd_master_500_kp_slave_500
 \labMasterSlaveSubFigure{PD}{P}{kp_master_500_kd_master_\kd_kp_slave_500}
    {K_{p_{\textrm{mestre}}}=500 \quad K_{d_{\textrm{mestre}}}=\kd \quad K_{p_{\textrm{escravo}}}=500 }%
}
\end{figure}

Como esperado a constante $K_d$ do controlador mestre, modifica o tempo de transição, overshooting e erro de regime do tanque 1,
no entanto, podemos observar que ela também influencia na transição, overshooting e erro de regime do nível do tanque 2.

\newpage

\subsubsection{Controlador Escravo PD}

\begin{figure}[h]
\foreach \kdMaster/\kdSlave in {1/1,1/100,5/100}{
    \labMasterSlaveSubFigure{PD}{PD}
    {Kp_master_500_Kd_master_\kdMaster_Kp_slave_500_Kd_slave_\kdSlave}
    {K_{p_{\textrm{mestre}}}=500 \quad K_{d_{\textrm{mestre}}}=\kdMaster
    \quad K_{p_{\textrm{escravo}}}=500 \quad K_{d_{\textrm{escravo}}}=\kdSlave}%
    }
\end{figure}

Como já esperado  modificações na constante $K_d$ tanto no mestre quando escravo influenciam no
tempo de transição, overshooting e erro de regime, tanto no nível do tanque 1 quando no 2.

\newpage

\subsubsection{Controlador Escravo PI}

\begin{figure}[h]
\foreach \kdMaster/\kiSlave in {1/10,1/50,10000/10}{
    \labMasterSlaveSubFigure{PD}{PI}
    {Kp_master_500_Kd_master_\kdMaster_Kp_slave_500_Ki_slave_\kiSlave}
    {K_{p_{\textrm{mestre}}}=500 \quad K_{d_{\textrm{mestre}}}=\kdMaster \quad K_{p_{\textrm{escravo}}}=500 \quad K_{i_{\textrm{escravo}}}=\kiSlave}%
    }
\end{figure}

analogamente ao controlador escravo PD, é observado que a componente integrativa, aumentar o overshooting e a variação dos níveis e
que com o $K_d$ do controlador Mestre é possível minimizar esses efeitos, mas trazendo os efeitos que uma componente derivativa trás,
como por exemplo o aumento no tempo de transição.

\newpage


\subsubsection{Controlador Escravo PID}

\begin{figure}[h]
\foreach  \kdMaster/\kiSlave/\kdSlave in {1/10/500,1/20/2000,2/20/2000}{
\labMasterSlaveSubFigure{PD}{PID}
{Kp_master_500_Kd_master_\kdMaster_kp_slave_500_ki_slave_\kiSlave_kd_slave_\kdSlave}
{K_{p_{\textrm{mestre}}}=500 \quad K_{d_{\textrm{mestre}}}=\kdMaster \quad K_{p_{\textrm{escravo}}}=500 \quad K_{i_{\textrm{escravo}}}=\kiSlave \quad K_{d_{\textrm{escravo}}}=\kdSlave}
}

\end{figure}

Como dito anteriormente, o controlador escravo influência fortemente no comportamento do nível do tanque 1,
que também acaba influênciando no comportamento do nível do tanque 2, com o acréscimo da componente derivativa
podemos observar que é possível influênciar ainda mais no comportamento do nível do tanque 2, no sentido de
reduzir o overshooting, erro de regime e aumentar o tempo de transição.

\newpage

\subsubsection{Controlador Escravo PID filtro derivativo}

\begin{figure}[h]
  \foreach \kdMaster/\ki/\kd/\kn in {1/20/2000/1.5,2/20/2000/1.5,2/20/2500/0.75}{
      \labMasterSlaveSubFigure{PD}{PID filtro derivativo}
    {Kp_master_500_Kd_master_\kdMaster_Kp_slave_500_Ki_slave_\ki_Kd_slave_\kd_Kn_slave_\kn}
    {K_{p_{\textrm{mestre}}}=500 \quad K_{d_{\textrm{master}}}=\kdMaster%
    K_{p_{\textrm{escravo}}}=500 \quad K_{i_{\textrm{escravo}}}=\ki%
    \quad K_{d_{\textrm{escravo}}}=\kd \quad K_{n_{\textrm{escravo}}}=\kn}
    }
\end{figure}
É observado que o filtro derivativo amplifica ainda mais os efeitos da componente derivativa do controlador
escravo e mestre, onde foi possível observar que o sistema passou mais de
duzentos segundos no regime transitório.

\newpage

\subsubsection{Controlador Escravo PID filtro windup}

\begin{figure}[h]
  \foreach \kdMaster/\ki/\kd/\kaw in {1/10/2/0.005,2.5/50/2.5/0.01,3/50/3/0.15}{
      \labMasterSlaveSubFigure{PD}{PID filtro windup}
    {Kp_master_500_Kd_master_\kdMaster_Kp_slave_500_Ki_slave_\ki_Kd_slave_\kd_Kaw_slave_\kaw}
    {K_{p_{\textrm{mestre}}}=500 \quad K_{d_{\textrm{mestre}}}=\kdMaster \quad %
     K_{p_{\textrm{escravo}}}=500 \quad K_{i_{\textrm{escravo}}}=\ki \quad  %
    K_{d_{\textrm{escravo}}}=\kd \quad K_{aw_{\textrm{escravo}}}=\kaw }
    }
\end{figure}

Escrever sobre o Controlador PID com filtro anti-reset-windup

\newpage
%
% 
\subsection{Sistema de Controle Mestre-Escravo: Mestre PI
}
%
\def \currentMaster{PI}
\def \currentSlave{escravo P}
\def \masterSlaveCaption{Mestre \currentMaster - \currentSlave }
\def \rootDir{mestre_PI}
%
\def \controllerPEvaluation{resultados/roteiro_c/\rootDir/Controlador_P}
\def \controllerPDEvaluation{resultados/roteiro_c/\rootDir/Controlador_PD}
\def \controllerPIEvaluation{resultados/roteiro_c/\rootDir/Controlador_PI}
\def \controllerPIDEvaluation{resultados/roteiro_c/\rootDir/Controlador_PID}
\def \controllerPIDDevvaluation{resultados/roteiro_c/\rootDir/Controlador_PID_filtro_dev}
\def \controllerPIDWindupEvaluation{resultados/roteiro_c/\rootDir/Controlador_PID_filtro_windup}
%
%
%
%
\subsubsection{Controlador Escravo P}
\begin{figure}[h]
    \labMasterSlaveSubFigure{PI}
{P}
{Kp_master_500_Ki_master_2_Kp_slave_500}
{K_{p_{\textrm{mestre}}}=500 \quad K_{i_{\textrm{mestre}}}=2 \quad K_{p_{\textrm{escravo}}}=500}
\labMasterSlaveSubFigure{PI}
{P}
{Kp_master_500_Ki_master_20_Kp_slave_500}
{K_{p_{\textrm{mestre}}}=500 \quad K_{i_{\textrm{mestre}}}=20 \quad K_{p_{\textrm{escravo}}}=500}
\labMasterSlaveSubFigure{PI}
{P}
{Kp_master_500_Ki_master_50_Kp_slave_500}
{K_{p_{\textrm{mestre}}}=500 \quad K_{i_{\textrm{mestre}}}=50 \quad K_{p_{\textrm{escravo}}}=500}

    \controllerCaption{1}{\masterSlaveCaption}
\end{figure}

\input{\controllerPEvaluation}

\newpage
%
\def \currentSlave{escravo PD}
%
\subsubsection{Controlador Escravo PD}

\begin{figure}[h]
    \labMasterSlaveSubFigure{PI}
{PD}
{Kp_master_500_Ki_master_0_5_Kp_slave_500_Kd_slave_10000}
{K_{p_{\textrm{mestre}}}=500 \quad K_{i_{\textrm{mestre}}}=0.5 \quad K_{p_{\textrm{escravo}}}=500 \quad K_{d_{\textrm{escravo}}}=10000}
\labMasterSlaveSubFigure{PI}
{PD}
{Kp_master_500_Ki_master_0_5_Kp_slave_500_Kd_slave_1000}
{K_{p_{\textrm{mestre}}}=500 \quad K_{i_{\textrm{mestre}}}=0.5 \quad K_{p_{\textrm{escravo}}}=500 \quad K_{d_{\textrm{escravo}}}=1000}
\labMasterSlaveSubFigure{PI}
{PD}
{Kp_master_500_Ki_master_2_Kp_slave_500_Kd_slave_300}
{K_{p_{\textrm{mestre}}}=500 \quad K_{i_{\textrm{mestre}}}=2 \quad K_{p_{\textrm{escravo}}}=500 \quad K_{d_{\textrm{escravo}}}=300}

    \controllerCaption{1}{\masterSlaveCaption}
\end{figure}

\input{\controllerPDEvaluation}

\newpage
%
\def \currentSlave{escravo PI}
%

\subsubsection{Controlador Escravo PI}

\begin{figure}[h]
    \labMasterSlaveSubFigure{PI}
{PI}
{Kp_master_500_Ki_master_0_5_Kp_slave_500_Ki_slave_0_5}
{K_{p_{\textrm{mestre}}}=500 \quad K_{i_{\textrm{mestre}}}=0.5 \quad K_{p_{\textrm{escravo}}}=500 \quad K_{i_{\textrm{escravo}}}=0.5}
\labMasterSlaveSubFigure{PI}
{PI}
{Kp_master_500_Ki_master_0_5_Kp_slave_500_Ki_slave_1}
{K_{p_{\textrm{mestre}}}=500 \quad K_{i_{\textrm{mestre}}}=0.5 \quad K_{p_{\textrm{escravo}}}=500 \quad K_{i_{\textrm{escravo}}}=1}
\labMasterSlaveSubFigure{PI}
{PI}
{Kp_master_500_Ki_master_0_5_Kp_slave_500_Ki_slave_5}
{K_{p_{\textrm{mestre}}}=500 \quad K_{i_{\textrm{mestre}}}=0.5 \quad K_{p_{\textrm{escravo}}}=500 \quad K_{i_{\textrm{escravo}}}=5}

    \controllerCaption{1}{\masterSlaveCaption}
\end{figure}

\input{\controllerPIEvaluation}

\newpage
%
\def \currentSlave{escravo PID}
%


\subsubsection{Controlador Escravo PID}

\begin{figure}[h]
    \labMasterSlaveSubFigure{PI}
{PID}
{Kp_master_500_Ki_master_0_5_Kp_slave_500_Ki_slave_0_5_Kd_slave_10000}
{K_{p_{\textrm{mestre}}}=500 \quad K_{i_{\textrm{mestre}}}=0.5 \quad K_{p_{\textrm{escravo}}}=500 \quad K_{i_{\textrm{escravo}}}=0.5 \quad K_{d_{\textrm{escravo}}}=10000}
\labMasterSlaveSubFigure{PI}
{PID}
{Kp_master_500_Ki_master_0_5_Kp_slave_500_Ki_slave_0_5_Kd_slave_1000}
{K_{p_{\textrm{mestre}}}=500 \quad K_{i_{\textrm{mestre}}}=0.5 \quad K_{p_{\textrm{escravo}}}=500 \quad K_{i_{\textrm{escravo}}}=0.5 \quad K_{d_{\textrm{escravo}}}=1000}
\labMasterSlaveSubFigure{PI}
{PID}
{Kp_master_500_Ki_master_0_5_Kp_slave_500_Ki_slave_0_5_Kd_slave_500}
{K_{p_{\textrm{mestre}}}=500 \quad K_{i_{\textrm{mestre}}}=0.5 \quad K_{p_{\textrm{escravo}}}=500 \quad K_{i_{\textrm{escravo}}}=0.5 \quad K_{d_{\textrm{escravo}}}=500}

    \controllerCaption{1}{\masterSlaveCaption}
\end{figure}

\input{\controllerPIDEvaluation}

\newpage
%
\def \currentSlave{escravo PID filtro derivativo}
%
\subsubsection{Controlador Escravo PID filtro derivativo}

\begin{figure}[h]
    \labMasterSlaveSubFigure{PI}
{PID filtro derivativo}
{Kp_master_500_Ki_master_2_Kp_slave_500_Ki_slave_0_5_Kd_slave_10000_Kn_slave_1}
{K_{p_{\textrm{mestre}}}=500 \quad K_{i_{\textrm{mestre}}}=2 \quad K_{p_{\textrm{escravo}}}=500 \quad K_{i_{\textrm{escravo}}}=0.5 \quad K_{d_{\textrm{escravo}}}=10000 \quad K_{n_{\textrm{escravo}}}=1}
\labMasterSlaveSubFigure{PI}
{PID filtro derivativo}
{Kp_master_500_Ki_master_2_Kp_slave_500_Ki_slave_0_5_Kd_slave_10000_Kn_slave_2}
{K_{p_{\textrm{mestre}}}=500 \quad K_{i_{\textrm{mestre}}}=2 \quad K_{p_{\textrm{escravo}}}=500 \quad K_{i_{\textrm{escravo}}}=0.5 \quad K_{d_{\textrm{escravo}}}=10000 \quad K_{n_{\textrm{escravo}}}=2}
\labMasterSlaveSubFigure{PI}
{PID filtro derivativo}
{Kp_master_500_Ki_master_30_Kp_slave_500_Ki_slave_0_5_Kd_slave_10000_Kn_slave_4_5}
{K_{p_{\textrm{mestre}}}=500 \quad K_{i_{\textrm{mestre}}}=30 \quad K_{p_{\textrm{escravo}}}=500 \quad K_{i_{\textrm{escravo}}}=0.5 \quad K_{d_{\textrm{escravo}}}=10000 \quad K_{n_{\textrm{escravo}}}=4.5}

    \controllerCaption{1}{\masterSlaveCaption}
\end{figure}
\input{\controllerPIDDevvaluation}

\newpage
%
\def \currentSlave{escravo PID filtro windup}
%
\subsubsection{Controlador Escravo PID filtro windup}

\begin{figure}[h]
    \labMasterSlaveSubFigure{PI}
{PID filtro windup}
{Kp_master_500_Ki_master_200_Kp_slave_500_Ki_slave_0_5_Kd_slave_45_Kaw_slave_2}
{K_{p_{\textrm{mestre}}}=500 \quad K_{i_{\textrm{mestre}}}=200 \quad K_{p_{\textrm{escravo}}}=500 \quad K_{i_{\textrm{escravo}}}=0.5 \quad K_{d_{\textrm{escravo}}}=45 \quad K_{aw_{\textrm{escravo}}}=2}
\labMasterSlaveSubFigure{PI}
{PID filtro windup}
{Kp_master_500_Ki_master_300_Kp_slave_500_Ki_slave_0_5_Kd_slave_45_Kaw_slave_2}
{K_{p_{\textrm{mestre}}}=500 \quad K_{i_{\textrm{mestre}}}=300 \quad K_{p_{\textrm{escravo}}}=500 \quad K_{i_{\textrm{escravo}}}=0.5 \quad K_{d_{\textrm{escravo}}}=45 \quad K_{aw_{\textrm{escravo}}}=2}
\labMasterSlaveSubFigure{PI}
{PID filtro windup}
{Kp_master_500_Ki_master_300_Kp_slave_500_Ki_slave_0_5_Kd_slave_45_Kaw_slave_5}
{K_{p_{\textrm{mestre}}}=500 \quad K_{i_{\textrm{mestre}}}=300 \quad K_{p_{\textrm{escravo}}}=500 \quad K_{i_{\textrm{escravo}}}=0.5 \quad K_{d_{\textrm{escravo}}}=45 \quad K_{aw_{\textrm{escravo}}}=5}

    \controllerCaption{1}{\masterSlaveCaption}
\end{figure}

\input{\controllerPIDWindupEvaluation}

\newpage
%
% 
\subsection{Sistema de Controle Mestre-Escravo: Mestre PID
}
%
\def \currentMaster{PID}
\def \currentSlave{escravo P}
\def \masterSlaveCaption{Mestre \currentMaster - \currentSlave }
\def \rootDir{mestre_PID}
%
\def \controllerPEvaluation{resultados/roteiro_c/\rootDir/Controlador_P}
\def \controllerPDEvaluation{resultados/roteiro_c/\rootDir/Controlador_PD}
\def \controllerPIEvaluation{resultados/roteiro_c/\rootDir/Controlador_PI}
\def \controllerPIDEvaluation{resultados/roteiro_c/\rootDir/Controlador_PID}
\def \controllerPIDDevvaluation{resultados/roteiro_c/\rootDir/Controlador_PID_filtro_dev}
\def \controllerPIDWindupEvaluation{resultados/roteiro_c/\rootDir/Controlador_PID_filtro_windup}
%
%
%
%
\subsubsection{Controlador Escravo P}
\begin{figure}[h]
    \labMasterSlaveSubFigure{PID}
{P}
{Kp_master_500_Ki_master_1_Kd_master_1_Kp_slave_500}
{K_{p_{\textrm{mestre}}}=500 \quad K_{i_{\textrm{mestre}}}=2 \quad K_{d_{\textrm{mestre}}}=1 \quad K_{p_{\textrm{escravo}}}=500}
\labMasterSlaveSubFigure{PID}
{P}
{Kp_master_500_Ki_master_1_Kd_master_500_Kp_slave_500}
{K_{p_{\textrm{mestre}}}=500 \quad K_{i_{\textrm{mestre}}}=1 \quad K_{d_{\textrm{mestre}}}=500 \quad K_{p_{\textrm{escravo}}}=500}
\labMasterSlaveSubFigure{PID}
{P}
{Kp_master_500_Ki_master_30_Kd_master_500_Kp_slave_500}
{K_{p_{\textrm{mestre}}}=500 \quad K_{i_{\textrm{mestre}}}=30 \quad K_{d_{\textrm{mestre}}}=500 \quad K_{p_{\textrm{escravo}}}=500}

    \controllerCaption{1}{\masterSlaveCaption}
\end{figure}

\input{\controllerPEvaluation}

\newpage
%
\def \currentSlave{escravo PD}
%
\subsubsection{Controlador Escravo PD}

\begin{figure}[h]
    \labMasterSlaveSubFigure{PID}
{PD}
{Kp_master_500_Ki_master_30_Kd_master_500_Kp_slave_500_Kd_slave_2}
{K_{p_{\textrm{mestre}}}=500 \quad K_{i_{\textrm{mestre}}}=30 \quad K_{d_{\textrm{mestre}}}=500 \quad K_{p_{\textrm{escravo}}}=500 \quad K_{d_{\textrm{escravo}}}=2}
\labMasterSlaveSubFigure{PID}
{PD}
{Kp_master_500_Ki_master_30_Kd_master_500_Kp_slave_500_Kd_slave_5}
{K_{p_{\textrm{mestre}}}=500 \quad K_{i_{\textrm{mestre}}}=30\quad K_{d_{\textrm{mestre}}}=500 \quad K_{p_{\textrm{escravo}}}=500 \quad K_{d_{\textrm{escravo}}}=2}
\labMasterSlaveSubFigure{PID}
{PD}
{Kp_master_500_Ki_master_30_Kd_master_500_Kp_slave_500_Kd_slave_10}
{K_{p_{\textrm{mestre}}}=500 \quad K_{i_{\textrm{mestre}}}=30 \quad K_{d_{\textrm{mestre}}}=500 \quad K_{p_{\textrm{escravo}}}=500 \quad K_{d_{\textrm{escravo}}}=5}

    \controllerCaption{1}{\masterSlaveCaption}
\end{figure}

\input{\controllerPDEvaluation}

\newpage
%
\def \currentSlave{escravo PI}
%

\subsubsection{Controlador Escravo PI}

\begin{figure}[h]
    \labMasterSlaveSubFigure{PID}
{PI}
{Kp_master_500_Ki_master_30_Kd_master_500_Kp_slave_500_Ki_slave_5}
{K_{p_{\textrm{mestre}}}=500 \quad K_{i_{\textrm{mestre}}}=30 \quad K_{d_{\textrm{mestre}}}=500 \quad K_{p_{\textrm{escravo}}}=500 \quad K_{i_{\textrm{escravo}}}=5}
\labMasterSlaveSubFigure{PID}
{PI}
{Kp_master_500_Ki_master_30_Kd_master_500_Kp_slave_500_Ki_slave_10}
{K_{p_{\textrm{mestre}}}=500 \quad K_{i_{\textrm{mestre}}}=30 \quad K_{d_{\textrm{mestre}}}=500 \quad K_{p_{\textrm{escravo}}}=500 \quad K_{i_{\textrm{escravo}}}=10}
\labMasterSlaveSubFigure{PID}
{PI}
{Kp_master_500_Ki_master_30_Kd_master_10000_Kp_slave_500_Ki_slave_0_5}
{K_{p_{\textrm{mestre}}}=500 \quad K_{i_{\textrm{mestre}}}=30 \quad K_{d_{\textrm{mestre}}}=10000 \quad K_{p_{\textrm{escravo}}}=500 \quad K_{i_{\textrm{escravo}}}=0.5}

    \controllerCaption{1}{\masterSlaveCaption}
\end{figure}

\input{\controllerPIEvaluation}

\newpage
%
\def \currentSlave{escravo PID}
%


\subsubsection{Controlador Escravo PID}

\begin{figure}[h]
    \labMasterSlaveSubFigure{PID}
{PID}
{Kp_master_500_Ki_master_30_Kd_master_500_Kp_slave_500_Ki_slave_0_5_Kd_slave_2}
{K_{p_{\textrm{mestre}}}=500 \quad K_{i_{\textrm{mestre}}}=30 \quad K_{d_{\textrm{mestre}}}=500 \quad K_{p_{\textrm{escravo}}}=500 \quad K_{i_{\textrm{escravo}}}=0.5 \quad K_{d_{\textrm{escravo}}}=2}
\labMasterSlaveSubFigure{PID}
{PID}
{Kp_master_500_Ki_master_30_Kd_master_500_Kp_slave_500_Ki_slave_0_5_Kd_slave_5}
{K_{p_{\textrm{mestre}}}=500 \quad K_{i_{\textrm{mestre}}}=30 \quad K_{d_{\textrm{mestre}}}=500 \quad K_{p_{\textrm{escravo}}}=500 \quad K_{i_{\textrm{escravo}}}=0.5 \quad K_{d_{\textrm{escravo}}}=5}
\labMasterSlaveSubFigure{PID}
{PID}
{Kp_master_500_Ki_master_30_Kd_master_10000_Kp_slave_500_Ki_slave_0_5_Kd_slave_1}
{K_{p_{\textrm{mestre}}}=500 \quad K_{i_{\textrm{mestre}}}=30 \quad K_{d_{\textrm{mestre}}}=10000 \quad K_{p_{\textrm{escravo}}}=500 \quad K_{i_{\textrm{escravo}}}=0.5 \quad K_{d_{\textrm{escravo}}}=1}

    \controllerCaption{1}{\masterSlaveCaption}
\end{figure}

\input{\controllerPIDEvaluation}

\newpage
%
\def \currentSlave{escravo PID filtro derivativo}
%
\subsubsection{Controlador Escravo PID filtro derivativo}

\begin{figure}[h]
    \labMasterSlaveSubFigure{PID}
{PID filtro derivativo}
{Kp_master_500_Ki_master_30_Kd_master_500_Kp_slave_500_Ki_slave_0_5_Kd_slave_5_Kn_slave_1}
{K_{p_{\textrm{mestre}}}=500 \quad K_{i_{\textrm{mestre}}}=30 \quad K_{d_{\textrm{mestre}}}=500 \quad K_{p_{\textrm{escravo}}}=500 \quad K_{i_{\textrm{escravo}}}=0.5 \quad K_{d_{\textrm{escravo}}}=10000 \quad K_{n_{\textrm{escravo}}}=1}
\labMasterSlaveSubFigure{PID}
{PID filtro derivativo}
{Kp_master_500_Ki_master_30_Kd_master_500_Kp_slave_500_Ki_slave_0_5_Kd_slave_5_Kn_slave_2}
{K_{p_{\textrm{mestre}}}=500 \quad K_{i_{\textrm{mestre}}}=30 \quad K_{d_{\textrm{mestre}}}=500 \quad K_{p_{\textrm{escravo}}}=500 \quad K_{i_{\textrm{escravo}}}=0.5 \quad K_{d_{\textrm{escravo}}}=10000 \quad K_{n_{\textrm{escravo}}}=2}
\labMasterSlaveSubFigure{PID}
{PID filtro derivativo}
{Kp_master_500_Ki_master_30_Kd_master_500_Kp_slave_500_Ki_slave_0_5_Kd_slave_5_Kn_slave_3}
{K_{p_{\textrm{mestre}}}=500 \quad K_{i_{\textrm{mestre}}}=30 \quad K_{d_{\textrm{mestre}}}=500 \quad K_{p_{\textrm{escravo}}}=500 \quad K_{i_{\textrm{escravo}}}=0.5 \quad K_{d_{\textrm{escravo}}}=10000 \quad K_{n_{\textrm{escravo}}}=3}

    \controllerCaption{1}{\masterSlaveCaption}
\end{figure}
\input{\controllerPIDDevvaluation}

\newpage
%
\def \currentSlave{escravo PID filtro windup}
%
\subsubsection{Controlador Escravo PID filtro windup}

\begin{figure}[h]
    \labMasterSlaveSubFigure{PID}
{PID filtro windup}
{Kp_master_500_Ki_master_5_Kd_master_500_Kp_slave_500_Ki_slave_5_Kd_slave_1_Kaw_slave_0_005}
{K_{p_{\textrm{mestre}}}=500 \quad K_{i_{\textrm{mestre}}}=5 \quad K_{d_{\textrm{mestre}}}=500 \quad K_{p_{\textrm{escravo}}}=500 \quad K_{i_{\textrm{escravo}}}=5 \quad K_{d_{\textrm{escravo}}}=1 \quad K_{aw_{\textrm{escravo}}}=0.005}
\labMasterSlaveSubFigure{PID}
{PID filtro windup}
{Kp_master_500_Ki_master_30_Kd_master_500_Kp_slave_500_Ki_slave_0_5_Kd_slave_5_Kaw_slave_0_005}
{K_{p_{\textrm{mestre}}}=500 \quad K_{i_{\textrm{mestre}}}=30 \quad K_{d_{\textrm{mestre}}}=500 \quad K_{p_{\textrm{escravo}}}=500 \quad K_{i_{\textrm{escravo}}}=0.5 \quad K_{d_{\textrm{escravo}}}=5 \quad K_{aw_{\textrm{escravo}}}=0.005}
\labMasterSlaveSubFigure{PID}
{PID filtro windup}
{Kp_master_500_Ki_master_30_Kd_master_500_Kp_slave_500_Ki_slave_30_Kd_slave_1_Kaw_slave_0_005}
{K_{p_{\textrm{mestre}}}=500 \quad K_{i_{\textrm{mestre}}}=30 \quad K_{d_{\textrm{mestre}}}=500 \quad K_{p_{\textrm{escravo}}}=500 \quad K_{i_{\textrm{escravo}}}=30 \quad K_{d_{\textrm{escravo}}}=1 \quad K_{aw_{\textrm{escravo}}}=0.005}

    \controllerCaption{1}{\masterSlaveCaption}
\end{figure}

\input{\controllerPIDWindupEvaluation}

\newpage
%
% 
\subsection{Sistema de Controle Mestre-Escravo: Mestre PID filtro derivativo
}
%
\def \currentMaster{PID filtro derivativo}
\def \currentSlave{escravo P}
\def \masterSlaveCaption{Mestre \currentMaster - \currentSlave }
\def \rootDir{mestre_PID_filtro_derivativo}
%
\def \controllerPEvaluation{resultados/roteiro_c/\rootDir/Controlador_P}
\def \controllerPDEvaluation{resultados/roteiro_c/\rootDir/Controlador_PD}
\def \controllerPIEvaluation{resultados/roteiro_c/\rootDir/Controlador_PI}
\def \controllerPIDEvaluation{resultados/roteiro_c/\rootDir/Controlador_PID}
\def \controllerPIDDevvaluation{resultados/roteiro_c/\rootDir/Controlador_PID_filtro_dev}
\def \controllerPIDWindupEvaluation{resultados/roteiro_c/\rootDir/Controlador_PID_filtro_windup}
%
%
%
%
\subsubsection{Controlador Escravo P}
\begin{figure}[h]
    \labMasterSlaveSubFigure{PID filtro derivativo}
{P}
{Kp_master_500_Ki_master_1_Kd_master_1_Kn_master_1_Kp_slave_500}
{K_{p_{\textrm{mestre}}}=500 \quad K_{i_{\textrm{mestre}}}=1 \quad K_{d_{\textrm{mestre}}}=1 \quad K_{n_{\textrm{mestre}}}=1 \quad K_{p_{\textrm{escravo}}}=500}
\labMasterSlaveSubFigure{PID filtro derivativo}
{P}
{Kp_master_500_Ki_master_7_Kd_master_1_Kn_master_200_Kp_slave_500}
{K_{p_{\textrm{mestre}}}=500 \quad K_{i_{\textrm{mestre}}}=7 \quad K_{d_{\textrm{mestre}}}=1 \quad K_{n_{\textrm{mestre}}}=200 \quad K_{p_{\textrm{escravo}}}=500}
\labMasterSlaveSubFigure{PID filtro derivativo}
{P}
{Kp_master_500_Ki_master_7_Kd_master_1_Kn_master_500_Kp_slave_500}
{K_{p_{\textrm{mestre}}}=500 \quad K_{i_{\textrm{mestre}}}=7 \quad K_{d_{\textrm{mestre}}}=1 \quad K_{n_{\textrm{mestre}}}=500 \quad K_{p_{\textrm{escravo}}}=500}

    \controllerCaption{1}{\masterSlaveCaption}
\end{figure}

\input{\controllerPEvaluation}

\newpage
%
\def \currentSlave{escravo PD}
%
\subsubsection{Controlador Escravo PD}

\begin{figure}[h]
    \labMasterSlaveSubFigure{PID filtro derivativo}
{PD}
{Kp_master_500_Ki_master_100_Kd_master_1_Kn_master_500_Kp_slave_500_Kd_slave_2}
{K_{p_{\textrm{mestre}}}=500 \quad K_{i_{\textrm{mestre}}}=100 \quad K_{d_{\textrm{mestre}}}=1 \quad K_{n_{\textrm{mestre}}}=500 \quad K_{p_{\textrm{escravo}}}=500 \quad K_{d_{\textrm{escravo}}}=2}
\labMasterSlaveSubFigure{PID filtro derivativo}
{PD}
{Kp_master_500_Ki_master_100_Kd_master_1_Kn_master_500_Kp_slave_500_Kd_slave_3}
{K_{p_{\textrm{mestre}}}=500 \quad K_{i_{\textrm{mestre}}}=100\quad K_{d_{\textrm{mestre}}}=1 \quad K_{n_{\textrm{mestre}}}=500 \quad K_{p_{\textrm{escravo}}}=500 \quad K_{d_{\textrm{escravo}}}=3}
\labMasterSlaveSubFigure{PID filtro derivativo}
{PD}
{Kp_master_500_Ki_master_100_Kd_master_1_Kn_master_500_Kp_slave_500_Kd_slave_5}
{K_{p_{\textrm{mestre}}}=500 \quad K_{i_{\textrm{mestre}}}=100 \quad K_{d_{\textrm{mestre}}}=1 \quad K_{n_{\textrm{mestre}}}=500 \quad K_{p_{\textrm{escravo}}}=500 \quad K_{d_{\textrm{escravo}}}=5}

    \controllerCaption{1}{\masterSlaveCaption}
\end{figure}

\input{\controllerPDEvaluation}

\newpage
%
\def \currentSlave{escravo PI}
%

\subsubsection{Controlador Escravo PI}

\begin{figure}[h]
    \labMasterSlaveSubFigure{PID filtro derivativo}
{PI}
{Kp_master_500_Ki_master_1_Kd_master_1_Kn_master_750_Kp_slave_500_Ki_slave_14}
{K_{p_{\textrm{mestre}}}=500 \quad K_{i_{\textrm{mestre}}}=1 \quad K_{d_{\textrm{mestre}}}=1 \quad K_{n_{\textrm{mestre}}}=750 \quad K_{p_{\textrm{escravo}}}=500 \quad K_{i_{\textrm{escravo}}}=14}
\labMasterSlaveSubFigure{PID filtro derivativo}
{PI}
{Kp_master_500_Ki_master_3_Kd_master_1_Kn_master_500_Kp_slave_500_Ki_slave_3}
{K_{p_{\textrm{mestre}}}=500 \quad K_{i_{\textrm{mestre}}}=3 \quad K_{d_{\textrm{mestre}}}=1 \quad K_{n_{\textrm{mestre}}}=500 \quad K_{p_{\textrm{escravo}}}=500 \quad K_{i_{\textrm{escravo}}}=3}
\labMasterSlaveSubFigure{PID filtro derivativo}
{PI}
{Kp_master_500_Ki_master_7_Kd_master_1_Kn_master_500_Kp_slave_500_Ki_slave_3}
{K_{p_{\textrm{mestre}}}=500 \quad K_{i_{\textrm{mestre}}}=7 \quad K_{d_{\textrm{mestre}}}=1 \quad K_{n_{\textrm{mestre}}}=500 \quad K_{p_{\textrm{escravo}}}=500 \quad K_{i_{\textrm{escravo}}}=3}

    \controllerCaption{1}{\masterSlaveCaption}
\end{figure}

\input{\controllerPIEvaluation}

\newpage
%
\def \currentSlave{escravo PID}
%


\subsubsection{Controlador Escravo PID}

\begin{figure}[h]
    \labMasterSlaveSubFigure{PID filtro derivativo}
{PID}
{Kp_master_2000_Ki_master_4_Kd_master_1_2_Kn_master_400_Kp_slave_2000_Ki_slave_4_Kd_slave_1}
{K_{p_{\textrm{mestre}}}=2000 \quad K_{i_{\textrm{mestre}}}=4 \quad K_{d_{\textrm{mestre}}}=1.2 \quad K_{n_{\textrm{mestre}}}=400 \quad K_{p_{\textrm{escravo}}}=2000 \quad K_{i_{\textrm{escravo}}}=4 \quad K_{d_{\textrm{escravo}}}=1}
\labMasterSlaveSubFigure{PID filtro derivativo}
{PID}
{Kp_master_2000_Ki_master_4_Kd_master_1_Kn_master_200_Kp_slave_2000_Ki_slave_4_Kd_slave_1}
{K_{p_{\textrm{mestre}}}=2000 \quad K_{i_{\textrm{mestre}}}=4 \quad K_{d_{\textrm{mestre}}}=1  \quad K_{n_{\textrm{mestre}}}=200\quad K_{p_{\textrm{escravo}}}=2000 \quad K_{i_{\textrm{escravo}}}=4 \quad K_{d_{\textrm{escravo}}}=1}
\labMasterSlaveSubFigure{PID filtro derivativo}
{PID}
{Kp_master_2000_Ki_master_4_Kd_master_1_Kn_master_400_Kp_slave_2000_Ki_slave_4_Kd_slave_1}
{K_{p_{\textrm{mestre}}}=2000 \quad K_{i_{\textrm{mestre}}}=4 \quad K_{d_{\textrm{mestre}}}=1 \quad K_{n_{\textrm{mestre}}}=400 \quad K_{p_{\textrm{escravo}}}=2000 \quad K_{i_{\textrm{escravo}}}=4 \quad K_{d_{\textrm{escravo}}}=1}

    \controllerCaption{1}{\masterSlaveCaption}
\end{figure}

\input{\controllerPIDEvaluation}

\newpage
%
\def \currentSlave{escravo PID filtro derivativo}
%
\subsubsection{Controlador Escravo PID filtro derivativo}

\begin{figure}[h]
    \labMasterSlaveSubFigure{PID filtro derivativo}
{PID filtro derivativo}
{Kp_master_2000_Ki_master_6_Kd_master_1_2_Kn_master_400_Kp_slave_2000_Ki_slave_4_Kd_slave_0_001_Kn_slave_2}
{K_{p_{\textrm{mestre}}}=2000 \quad K_{i_{\textrm{mestre}}}=6 \quad K_{d_{\textrm{mestre}}}=1.2  \quad K_{n_{\textrm{mestre}}}=400  \quad K_{p_{\textrm{escravo}}}=2000 \quad K_{i_{\textrm{escravo}}}=4 \quad K_{d_{\textrm{escravo}}}=0.001 \quad K_{n_{\textrm{escravo}}}=2}
\labMasterSlaveSubFigure{PID filtro derivativo}
{PID filtro derivativo}
{Kp_master_2000_Ki_master_6_Kd_master_1_2_Kn_master_400_Kp_slave_2000_Ki_slave_6_Kd_slave_0_01_Kn_slave_700}
{K_{p_{\textrm{mestre}}}=2000 \quad K_{i_{\textrm{mestre}}}=6 \quad K_{d_{\textrm{mestre}}}=1.2 \quad K_{n_{\textrm{mestre}}}=400  \quad K_{p_{\textrm{escravo}}}=2000 \quad K_{i_{\textrm{escravo}}}=6 \quad K_{d_{\textrm{escravo}}}=0.01 \quad K_{n_{\textrm{escravo}}}=700}
\labMasterSlaveSubFigure{PID filtro derivativo}
{PID filtro derivativo}
{Kp_master_2000_Ki_master_6_Kd_master_2_5_Kn_master_400_Kp_slave_2000_Ki_slave_6_Kd_slave_0_01_Kn_slave_700}
{K_{p_{\textrm{mestre}}}=2000 \quad K_{i_{\textrm{mestre}}}=6 \quad K_{d_{\textrm{mestre}}}=2.5 \quad K_{n_{\textrm{mestre}}}=400  \quad K_{p_{\textrm{escravo}}}=2000 \quad K_{i_{\textrm{escravo}}}=6 \quad K_{d_{\textrm{escravo}}}=0.01 \quad K_{n_{\textrm{escravo}}}=700}

    \controllerCaption{1}{\masterSlaveCaption}
\end{figure}
\input{\controllerPIDDevvaluation}

\newpage
%
\def \currentSlave{escravo PID filtro windup}
%
\subsubsection{Controlador Escravo PID filtro windup}

\begin{figure}[h]
    \labMasterSlaveSubFigure{PID filtro derivativo}
{PID filtro windup}
{Kp_master_200_Ki_master_1_Kd_master_0_01_Kp_slave_200_Ki_slave_2_Kd_slave_100_Kaw_slave_0_002}
{K_{p_{\textrm{mestre}}}=200 \quad K_{i_{\textrm{mestre}}}=1 \quad K_{d_{\textrm{mestre}}}=0.01 \quad K_{p_{\textrm{escravo}}}=200 \quad K_{i_{\textrm{escravo}}}=2 \quad K_{d_{\textrm{escravo}}}=100 \quad K_{aw_{\textrm{escravo}}}=0.002}
\labMasterSlaveSubFigure{PID filtro derivativo}
{PID filtro windup}
{Kp_master_200_Ki_master_1_Kd_master_0_05_Kp_slave_200_Ki_slave_1_Kd_slave_1_Kaw_slave_0_001}
{K_{p_{\textrm{mestre}}}=200 \quad K_{i_{\textrm{mestre}}}=1 \quad K_{d_{\textrm{mestre}}}=0.05 \quad K_{p_{\textrm{escravo}}}=200 \quad K_{i_{\textrm{escravo}}}=1 \quad K_{d_{\textrm{escravo}}}=1 \quad K_{aw_{\textrm{escravo}}}=0.001}
\labMasterSlaveSubFigure{PID filtro derivativo}
{PID filtro windup}
{Kp_master_200_Ki_master_1_Kd_master_0_05_Kp_slave_200_Ki_slave_1_Kd_slave_700_Kaw_slave_0_0001}
{K_{p_{\textrm{mestre}}}=200 \quad K_{i_{\textrm{mestre}}}=1 \quad K_{d_{\textrm{mestre}}}=0.05 \quad K_{p_{\textrm{escravo}}}=200 \quad K_{i_{\textrm{escravo}}}=1 \quad K_{d_{\textrm{escravo}}}=700 \quad K_{aw_{\textrm{escravo}}}=0.0001}

    \controllerCaption{1}{\masterSlaveCaption}
\end{figure}

\input{\controllerPIDWindupEvaluation}
