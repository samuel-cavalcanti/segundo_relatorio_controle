Quando o sistema está sujeito a um erro constante diferente de zero, o controlador integral tende a acumular seu efeito e requisitar comandos cada vez maiores ao atuador, mesmo estando saturado. Quando o erro eventualmente decresce e se torna negativo, há um tempo até que a saída integral “retorne” ao seu estado funcional, é o que se chama de tempo de \emph{windup}. O filtro \emph{anti-windup} contrabalanceia esse efeito somando ou subtraindo um valor proporcional ao ganho $K_aw$ quando há saturação (saída do controlador e entrada da planta são diferentes).

O filtro \emph{anti-windup} atenua o efeito da parcela integral do controlador. Com $K_aw$ = 0.05 há uma leve redução do \emph{overshoot} e do tempo de acomodação. A frequência natural não foi modificada ao aumentar $K_aw$, apenas a mesma redução citada anteriormente, \emph{overshoot} por exemplo cai de 50\% para 35\% e tempo de acomodação (aproximadamente 5\%) de 180 ms para 130 ms aumentando o ganho do filtro de 0.05 para 0.5. O \emph{overshoot} do tanque 2 é completamente removido para valores menores de $K_i$.

