Quando o sistema está sujeito a um erro constante diferente de zero, o controlador integral tende a acumular seu efeito e requisitar comandos cada vez maiores ao atuador, mesmo estando saturado. Quando o erro eventualmente decresce e se torna negativo, há um tempo até que a saída integral “retorne” ao seu estado funcional, é o que se chama de tempo de \emph{windup}. O filtro \emph{anti-windup} contrabalanceia esse efeito somando ou subtraindo um valor proporcional ao ganho $K_aw$ quando há saturação (saída do controlador e entrada da planta são diferentes).

No primeiro sistema, o filtro \emph{anti-windup} atenua o efeito da parcela integral do controlador. Com $K_aw$ = 1 há uma leve redução do \emph{overshoot} e do tempo de acomodação. A frequência natural não foi modificada ao aumentar $K_aw$, apenas a mesma redução citada anteriormente, \emph{overshoot} por exemplo cai de 12\% para 9\%  aumentando o ganho do filtro de $K_aw$ 0.5 para 1. O \emph{overshoot} do tanque 2 é completamente removido para valores menores de $K_i$.

No segundo sistema, mesmo com $K_i$ = 1 a oscilação é controlada até que se aumente seu valor para algo mais considerável (no caso 5), onde há \emph{overshoot}. Aumentar $K_d$ tem efeito de redução da amplitude das oscilações, como esperado.

