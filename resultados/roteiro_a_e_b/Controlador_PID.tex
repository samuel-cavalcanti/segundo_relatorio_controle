
Nesse contexto, o controlador proporcional ajusta a variável de controle proporcionalmente ao erro, o integral ajusta proporcional ao erro acumulado e a derivativa proporcional a velocidade de variação do erro. Permite um maior controle do comportamento do sistema, em relação aos anteriores. Observou-se um aumento no \emph{overshoot} ao se aumentar $K_i$ e uma maior oscilação com aumento de $K_p$.

Como observado no controlador PI, instabilização ocorre para maiores valores de $K_i$, onde o sistema nunca alcança regime permanente. O mesmo efeito ocorre ao se alterar $K_p$ mas não na mesma proporção. Alterar $K_d$ afeta notavelmente o tempo de acomodação e a sensibilidade do controlador ao efeito do ruído, como visto no do tipo PD. Para valores menores de $K_d$ (notadamente $K_d$ = 100) a frequência diminui de tal maneira a aumentar o tempo de acomodação. Para entrada senoidal, foi possível ajustar $K_i$ e $K_d$ para mitigar os efeitos de oscilação no nível do primeiro tanque ao mesmo tempo que se reduz a velocidade de regime transitório. 

