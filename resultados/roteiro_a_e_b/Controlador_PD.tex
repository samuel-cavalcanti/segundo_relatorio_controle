
A ação derivativa nesse contexto tem função de antecipar a ação de controle, ao levar em conta a “inércia” do erro, gerando maior estabilidade relativa e velocidade de resposta transitória. Com isso se observou o aumento do erro relativo a medida em que se aumentada o ganho diferencial $K_d$. Também é atestado a amplificação do ruído, já que o mesmo é formado por bruscas variações em curtos períodos de tempo. Esse ruído amplificado, no entanto, é saturado em -4/+4. 

Para entrada senoidal no sistema 1 o bias de referência foi afetado: a medida em que se aumentava $K_d$ a média permanente do sinal diminuía em relação a referencia. No caso, de 10 para 9 em $K_d$ = 100.

Já no sistema 2, ao aumentar o ganho diferencial $K_d$ o \emph{overshoot} foi atenuado em aproximadamente 10\% de $K_d$ = 1 para $K_d$ = 25, enquanto a sensibilidade do controlador as alterações advindas do ruído foram bastante acentuadas. Além disso, nesse contexto, a velocidade da resposta transitória foi praticamente dobrada (tempo de acomodação cai pela metade). Na entrada senoidal o aumento do $K_d$ reduziu oscilações no nível do tanque 1 e \emph{overshoot} inicial, além disso, fez o nível do 2 seguir melhor a referência.
