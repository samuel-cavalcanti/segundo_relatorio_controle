Testes de controle de primeira ordem, para entrada constante de valor 15 e senoidal, de amplitude 2 e bias 10.

Pode-se observar que, no sistema de tipo 1, com o aumento do ganho proporcional houve um
aumento na velocidade do sistema até a resposta em regime permanente e
uma diminuição do erro estacionário, ou seja,
a diferença entre a resposta desejada e a resposta em regime permanente. No senoidal ocorreu o mesmo, aumentando $K_p$ o nível do tanque 1 passa a seguir a referência e o nível do segundo passa a orbitar o bias, sem nunca estabilizar por causa da baixa frequência.

Observou-se que também que, no sistema de tipo 2,
o aumento de $K_p$ aumentou a oscilação do sistema mas diminuiu o 
erro de regime, não zerando-o completamente, como esperado. Sendo assim,
o sistema é do tipo subamortecido (0 < $\epsilon$ < 1), e considerando
que $K_p$ = $\omega_n^2$, a frequência aumenta com o ganho. Houve aumento
no tempo de acomodação e haveria também aumento significativo no
\emph{overshoot}, como se pode observar pela resposta do controlador
que é quase uma onda quadrada em $K_p$ = 100.
Não ocorreu devido a saturação do sistema. Com entrada senoidal, o aumento do $K_p$ resultou na resposta do segundo tanque seguir exatamente a referência e oscilações no nível do primeiro tanque.